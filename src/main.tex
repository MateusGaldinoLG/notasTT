\documentclass{article}

\usepackage[utf8]{inputenc}
\usepackage{subfiles}
\usepackage{bookmark}
\usepackage{amsthm}
\usepackage{mathtools}
\usepackage{amssymb}

\theoremstyle{definition}
\newtheorem{definition}{Definição}[section]
\newtheorem{lemma}{Lemma}[section]
\newtheorem{theorem}{Teorema}[section]
\newtheorem{corollary}{Corolário}[section]

\title{Notas em Teoria dos Tipos}
\author{Mateus Galdino}
\date{Recife, 2023}

\begin{document}
    \maketitle

    \tableofcontents

    \newpage

        \subfile{sections/prefacio}

    \newpage

        % PARTE 1

        \part{O Cubo Lambda}
            \setcounter{section}{0}
            \renewcommand*{\theHsection}{chX.\the\value{section}}

            \subfile{sections/parte1/lambda.tex}
            \newpage
            \subfile{sections/parte1/simple.tex}

    \newpage
        % PARTE 2

        \part{Construções paralelas ao cubo}

    \newpage
        % PARTE 3

        \part{Semântica Categorial das teorias do cubo lambda}

    \newpage
        % PARTE 4

        \part{Teorias Homotópicas de Tipos}


    \newpage
        % PARTE 5
        \part{Semântica Categorial das teoria homotópicas de tipos}

    \newpage   
        % PARTE 6
        \part{Lógica}
    \newpage

    \newpage
\end{document}

\documentclass[a4paper]{article}

\usepackage[utf8]{inputenc}
\usepackage{subfiles}
\usepackage{bookmark}
\usepackage{amsthm}
\usepackage{mathtools}
\usepackage{amssymb}
\usepackage{bussproofs}
% \usepackage{pagestyle/fitch}
\usepackage{mathpartir}
\usepackage{stmaryrd}
\usepackage[useregional]{datetime2}

\usepackage[alf, abnt-repeated-title-omit=yes, abnt-emphasize=bf, abnt-etal-list=0]{abntex2cite}
\citebrackets()

\theoremstyle{definition}
\newtheorem{definition}{Definição}[section]
\newtheorem{lemma}{Lema}[section]
\newtheorem{theorem}{Teorema}[section]
\newtheorem{corollary}{Corolário}[section]
\newtheorem{proposition}{Proposição}[section]

\begin{document}
    \begin{titlepage}
        \begin{center}
            \vspace*{1cm}
    
            \Huge
            \textbf{Tipos, Categorias e Lógicas}
    
            \vspace{0.5cm}
            \large
            Notas em Teoria dos Tipos, Teoria das Categorias e Lógica.
                  
            \vfill



            \Large
            edição \textbf{$0.0$}

            \textbf{Autores: Mateus Galdino}

            \today
                
        \end{center}
    \end{titlepage}

    \begingroup
        \hypersetup{hidelinks}    
        \tableofcontents

    \endgroup


    \newpage

        \subfile{sections/prefacio}

    \newpage

        % PARTE 1

        \part{O Cubo Lambda}
            \setcounter{section}{0}
            \renewcommand*{\theHsection}{chX.\the\value{section}}

            \subfile{sections/parte1/lambda.tex}
            \newpage
            \subfile{sections/parte1/simple.tex}
            \newpage
            \subfile{sections/parte1/systemF.tex}
            \newpage
            \subfile{sections/parte1/omega.tex}

    \newpage
        % PARTE 2

        \part{Construções paralelas ao cubo}

    \newpage
        % PARTE 3

        \part{Semântica Categorial das teorias do cubo lambda}
            \setcounter{section}{0}
            \renewcommand*{\theHsection}{chX.\the\value{section}}

            \subfile{sections/parte3/cattheory.tex}
    \newpage
        % PARTE 4

        \part{Teorias Homotópicas de Tipos}


    \newpage
        % PARTE 5
        \part{Semântica Categorial das teoria homotópicas de tipos}

    \newpage   
        % PARTE 6
        \part{Lógica}
    \newpage

    \newpage
        % Apêndice Histórico
        \part{Apêndices}
        \setcounter{section}{0}

            \subfile{sections/parte7/apendice.tex}

    \newpage
        % BIBLIOGRAFIA
        \bibliography{referencias}

    \newpage
\end{document}

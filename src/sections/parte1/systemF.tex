\documentclass[../main.tex]{subfiles}

\begin{document}

\section{O Sistema F}

No Cálculo-Lambda Simplesmente Tipado, é possível definir a função identidade, a função que pega um valor como input e retorna o próprio valor como outpu, para cada tipo definido no cálculo:

\begin{itemize}
    \item Para os números naturais, $\lambda x : \mathbb{N} . x$
    \item Para os booleanos, $\lambda x: bool . x$
    \item Para o tipo das funções dos naturais nos booleanos, $\lambda x : (\mathbb{N} \to bool) . x$
    \item $\dots$
\end{itemize}

Mas dessa forma, quanto mais tipos a teoria suportar, mais formais diferentes são possíveis de serem criadas. Isso faz com que existam vários termos análogos sem qualquer possibilidade de relação entre eles. O máximo que se pode dizer é fazer uma quantificação além de $\lambda_{\to}$ e construir um tipo arbitrário $\alpha$ com uma função $f \equiv \lambda x : \alpha . x$ que seria a função identidade arbitrária.

Porém, dado um termo $M : \mathbb{N}$, não é possível escrever $fM$ pois $\alpha \not\equiv \mathbb{N}$. Para fazer isso, é necessário que a função receba também o tipo específico que ela precisa ter para receber o termo $M$, fazendo um segundo processo de abstração em cima da função da seguinte forma: $$\lambda \alpha : \ast . \lambda x : \alpha . x$$

Nesse caso, $\alpha$ se torna uma variável de tipo e $\star$ o tipo de todos os tipos. Esse termo é chamado de \emph{polimórfico}, pois pode possuir diversas formas diferentes a depender do tipo escolhido:

\begin{itemize}
    \item $(\lambda \alpha : \ast . \lambda x : \alpha . x)\mathbb{N} \to_{\beta} \lambda x : \mathbb{N} . x$
\end{itemize}

Para fazer essa extensão, é necessário adicionar regras de inferência e regras de tipagem que lidem com essa abstração de segunda ordem.

A tipagem para a função identidade $\lambda \alpha : \ast . \lambda x : \alpha . x$ é o tipo $\Pi \alpha : \ast . \alpha \to \alpha$, onde $\Pi$ é o operador que tem como função ligar os tipos, chamado de Tipo $\Pi$ ou Tipo Produto

Exemplos:

\begin{itemize}
    \item A função de iteração $D$ que recebe uma função $f : \alpha \to \alpha$ e retorna a aplicação dela duas vezes em cima de um termo $x : \alpha$ pode ser escrita da seguinte forma: $$D \equiv \lambda \alpha : \ast . \lambda f : \alpha \to \alpha . \lambda x : \alpha . f (f x)$$
    Nesse caso, $D$ é a mesma coisa que $f \circ f$.
    Para os números naturais: $$D \mathbb{N} \equiv \lambda f : \mathbb{N} \to \mathbb{N} . \lambda x : \mathbb{N} . f (f x)$$
    e sendo $f$ a função sucessor $s$ que mapeia $n : \mathbb{N}$ em $n + 1 : \mathbb{N}$, então: $$D \mathbb{N} s \to_{\beta} \lambda x : \mathbb{N} . s (s x)$$
    O tipo de $D$ é: $D : \Pi \alpha : \ast . (\alpha \to \alpha) \to \alpha \to \alpha$
    \item A composição de duas funções é a aplicação de uma função em outra. É possível definir o operador de composição $\circ$ da seguinte forma: $$\circ \equiv \lambda \alpha : \ast . \lambda \beta : \ast \lambda \gamma : \ast . \lambda f : \alpha \to \beta . \lambda g : \beta \to \gamma . \lambda x : \alpha . g (f x)$$
    A sua tipagem é: $\circ : \Pi \alpha : \ast . \Pi \beta : \ast . \Pi \gamma : \ast . (\alpha \to \beta) \to (\beta \to \gamma) \to \alpha \to \gamma$
\end{itemize}


\subsection{O Cálculo Lambda com tipagem de Segunda Ordem}

\subsubsection{Regras de Inferência}

Uma vez inseridas as regras de abstração e aplicação de segunda ordem, é necessário extender as regras de inferência em relação ao ST$\lambda$C


\begin{definition}[Regra de Inferência para a Abstração]
    \hfil
    \begin{prooftree}
        \def\fCenter{\mbox{\ $\vdash$\ }}
        \AxiomC{$\Gamma, \alpha : \ast \vdash M : A$}
        \RightLabel{$abst_2$}
        \UnaryInfC{$\Gamma \vdash \lambda \alpha : \ast . M : \Pi \alpha : \ast . A$}
    \end{prooftree}
\end{definition}

Essa regra define basicamente que, sendo $M$ um termo de tipo $A$ em um contexto onde $\alpha$ possui tipo $\ast$, então a abstração $\alpha : \ast . M$ possui o tipo $\Pi \alpha : \ast . A$. Essa regra da abstração difere da primeira por permitir a definição de $\alpha$ no contexto.

\begin{definition}[Regra de Inferência para a Aplicação]
    \hfil
    \begin{prooftree}
        \def\fCenter{\mbox{\ $\vdash$\ }}
        \AxiomC{$\Gamma \vdash M : \Pi \alpha : \ast . A$}
        \AxiomC{$\Gamma \vdash B : \ast$}
        \RightLabel{$appl_2$}
        \BinaryInfC{$\Gamma \vdash MB : A[\alpha := B]$}
    \end{prooftree}
\end{definition}

\subsubsection[O Sistema Lambda2]{O Sistema $\lambda 2$}

A sintaxe de $\lambda2$ segue de forma análoga a $\lambda_{\sigma}$, sendo descrita pela seguinte BNF: $$\mathbb{T}2 = \mathbb{V} | (\mathbb{T}2 \to \mathbb{T}2) | (\Pi\mathbb{V} : \ast . \mathbb{T}2) $$

onde $\mathbb{V}$ é a coleção dos tipos variáveis, denominados de $\alpha, \beta, \gamma, \dots$.

Para os termos pré-tipados:

\begin{definition}
    A coleção dos $\lambda$-termos pré-tipados de segunda ordem, ou $\lambda2$-termos, é definido na seguinte BNF: $$\Lambda_{\mathbb{T}2} = V | (\Lambda_{\mathbb{T}2} \Lambda_{\mathbb{T}2}) | (\Lambda_{\mathbb{T}2} {\mathbb{T}2}) | (\lambda V : \mathbb{T}2 . \Lambda_{\mathbb{T}2}) | (\lambda \mathbb{V} : \ast . \Lambda_{\mathbb{T}2})$$
\end{definition}

Onde $V$ é a coleção das variáveis de termos ($x, y, z, \dots$). Como existem ambos $\mathbb{V}$ e $V$, então a BNF possui duas formas de aplicação, uma de primeira ordem ($\lambda V : \mathbb{T}2 . \Lambda_{\mathbb{T}2}$) para variáveis de termo e outro de segunda ordem ($\lambda \mathbb{V} : \ast . \Lambda_{\mathbb{T}2}$) para variáveis de tipo.

Da mesma forma, também existe a aplicação de primeira ordem ($\Lambda_{\mathbb{T}2} \Lambda_{\mathbb{T}2}$) e de segunda ordem ($\Lambda_{\mathbb{T}2} {\mathbb{T}2}$).

As regras de parenteses em aplicação e abstração segue as regras vistas anteriormente para o ST$\lambda$C e para o $\lambda_{\beta\eta}$:

\begin{itemize}
    \item Parenteses mais externos podem ser omitidos
    \item Aplicação é associativa à esquerda
    \item Aplicação e $\to$ precedem ambas abstrações $\lambda$ e $\Pi$
    \item Abstrações $\lambda$ e $\Pi$ sucessivas com o mesmo tipo podem ser combinadas de forma associativa à direita
    \item Tipos funcionais são escritos de forma associativa à direita
\end{itemize}

Exemplo: $(\Pi \alpha : \ast . (\Pi \beta : \ast . (\alpha \to (\beta \to \alpha))))$ pode ser escrito como $\Pi \alpha, \beta : \ast . \alpha \to \beta \to \alpha$.

A definição para declarações e sentenças pode ser estendida da seguinte forma:

\begin{definition}[Declarações, sentenças]
    \hfil
    \begin{itemize}
        \item Uma \emph{sentença} possui a forma $M : \sigma$ onde $M \in \Lambda_{\mathbb{T}2}$ e $\sigma \in \mathbb{T}2$ ou da forma $\sigma : \ast$, onde $\sigma \in \mathbb{T}2$
        \item Uma \emph{declaração} é uma sentença com uma variável de termo ou uma variável de tipo como sujeito
    \end{itemize}
\end{definition}

Para $\lambda2$ como é possível que uma variável de termo faça uso de uma variável de tipo, é necessário que a ordem da aparição dessas variáveis siga uma regra, para que uma variável não seja usada antes de ser declarada. O contexto pode ser descrito como um \emph{domínio} da seguinte forma:

\begin{definition}[Contexto de $\lambda2$]
    \hfil
    \begin{enumerate}
        \item $\emptyset$ é um contexto válido de $\lambda2$ \\
        $dom(\emptyset) = ()$, a lista vazia
        \item Se $\Gamma$ for um contexto de $\lambda2$, $\alpha \in \mathbb{V}$ e $\alpha \not\in dom(\Gamma$), então $\Gamma, \alpha : \ast$ é um contexto de $\lambda2$ \\
        $dom(\Gamma, \alpha : \ast) = (dom(\Gamma), \alpha)$, ou seja $dom(\Gamma)$ concatenado com $\alpha$
        \item Se $\Gamma$ for um contexto de $\lambda2$, se $\rho \in \mathbb{T}2$ tal que $\alpha \in dom(\Gamma)$ para toda variável de tipo livre $\alpha$ existente em $\rho$ e se $x \not\in dom(\Gamma)$, então $\Gamma, x : \rho$ é um contexto de $\lambda2$ \\
        $dom(\Gamma, x : \rho) = (dom(\Gamma), x)$
    \end{enumerate}
\end{definition}

\textbf{Exemplos}

\begin{itemize}
    \item $\emptyset$ é um contexto de $\lambda2$ por (1)
    \item $\alpha : \ast$ é um contexto de $\lambda2$ por (2)
    \item $\alpha : \ast, x : \alpha \to \alpha$ é um contexto de $\lambda2$ por (3)
    \item logo $\alpha : \ast, x : \alpha \to \alpha, \beta : \ast$ é um contexto de $\lambda2$ por (2)
    \item e $\alpha : \ast, x : \alpha \to \alpha, \beta : \ast, y : (\alpha \to \alpha) \to \beta$ é um contexto de $\lambda2$ por (3), sendo $dom(\Gamma) = (\alpha, x, \beta, y)$
\end{itemize}

A regra \emph{var} pode ser reconstruida para lidar com os tipos de $\lambda2$:

\begin{definition} (Regra var em $\lambda2$)
    \hfil
    (\emph{var}) $\Gamma \vdash x : \sigma$ se $\Gamma$ for um contexto de $\lambda2$ e $x : \sigma \in \Gamma$
\end{definition}

O problema é que, usando as regras até então, não é possível chegar ao juizo $\Gamma \vdash B : \ast$. Por isso, será introduzida uma nova regra:

\begin{definition}(Regra de formação)
    \hfil
    (\emph{form}) $\Gamma \vdash B : \ast$ se $\Gamma$ é um contexto de $\lambda2$, $B \in \mathbb{T}2$ e todas as variáveis de tipo livres em $B$ sejam declaradas em $\Gamma$
\end{definition}

Regras de $\lambda2$:

\begin{itemize}
    \item (\emph{var}) $\Gamma \vdash x : \sigma$ se $\Gamma$ for um contexto de $\lambda2$ e $x : \sigma \in \Gamma$
    \item (\emph{appl}) \begin{prooftree}
        \def\fCenter{\mbox{\ $\vdash$\ }}
        \AxiomC{$\Gamma \vdash M : \sigma \to \tau$}
        \AxiomC{$\Gamma \vdash N : \sigma$}
        \RightLabel{appl}
        \BinaryInfC{$\Gamma \vdash MN : \tau$}
    \end{prooftree}
    \item (\emph{abst}) \begin{prooftree}
        \def\fCenter{\mbox{\ $\vdash$\ }}
        \AxiomC{$\Gamma, x : \sigma \vdash M : \tau$}
        \RightLabel{abst}
        \UnaryInfC{$\Gamma \vdash \lambda x : \sigma . M : \sigma \to \tau$}
    \end{prooftree}
    \item (\emph{form}) $\Gamma \vdash B : \ast$ se $\Gamma$ é um contexto de $\lambda2$, $B \in \mathbb{T}2$ e todas as variáveis de tipo livres em $B$ sejam declaradas em $\Gamma$
    \item ($appl_2$) \begin{prooftree}
        \def\fCenter{\mbox{\ $\vdash$\ }}
        \AxiomC{$\Gamma \vdash M : \Pi \alpha : \ast . A$}
        \AxiomC{$\Gamma \vdash B : \ast$}
        \RightLabel{$appl_2$}
        \BinaryInfC{$\Gamma \vdash MB : A[\alpha := B]$}
    \end{prooftree}
    \item ($abst_2$) \begin{prooftree}
        \def\fCenter{\mbox{\ $\vdash$\ }}
        \AxiomC{$\Gamma, \alpha : \ast \vdash M : A$}
        \RightLabel{$abst_2$}
        \UnaryInfC{$\Gamma \vdash \lambda \alpha : \ast . M : \Pi \alpha : \ast . A$}
    \end{prooftree}
\end{itemize}

\begin{definition}($\lambda2$-termos legais)
    Um termo $M$ em $\Lambda_{\mathbb{T}2}$ é chamado de \emph{legal} se existe um contexto de $\lambda2$ $\Gamma$ e um tipo $\rho$ em $\mathbb{T}2$ tal que $\Gamma \vdash M : \rho$    
\end{definition}

\subsubsection{Exemplos de Derivação}

Seja a seguinte árvore de inferência incompleta:

\begin{prooftree}
    \def\fCenter{\mbox{\ $\vdash$\ }}
    \AxiomC{?}
    \RightLabel{$?$}
    \UnaryInfC{$\emptyset \vdash \lambda \alpha : \ast . \lambda f : \alpha \to \alpha . \lambda x : \alpha . f (fx) : \Pi \alpha : \ast . (\alpha \to \alpha) \to \alpha \to \alpha$}
\end{prooftree}

Primeiro, é necessário utilizar a regra ($abst_2$):

\begin{prooftree}
    \def\fCenter{\mbox{\ $\vdash$\ }}
    \AxiomC{$?$}
    \RightLabel{$?$}
    \UnaryInfC{$\alpha : \ast \vdash \lambda f : \alpha \to \alpha . \lambda x : \alpha . f (fx) : (\alpha \to \alpha) \to \alpha \to \alpha$}
    \RightLabel{$abst_2$}
    \UnaryInfC{$\emptyset \vdash \lambda \alpha : \ast . \lambda f : \alpha \to \alpha . \lambda x : \alpha . f (fx) : \Pi \alpha : \ast . (\alpha \to \alpha) \to \alpha \to \alpha$}
\end{prooftree}

Após isso as regras que precisam ser utilizadas já são conhecidas a partir do ST$\lambda$C:

primeiro dois absts seguidos para $f$ e $x$:

\begin{prooftree}
    \def\fCenter{\mbox{\ $\vdash$\ }}
    \AxiomC{$?$}
    \RightLabel{$?$}
    \UnaryInfC{$\alpha : \ast, f : \alpha \to \alpha, x : \alpha \vdash f (fx) : \alpha$}
    \RightLabel{$abst$}
    \UnaryInfC{$\alpha : \ast, f : \alpha \to \alpha \vdash \lambda x : \alpha . f (fx) : \alpha \to \alpha$}
    \RightLabel{$abst$}
    \UnaryInfC{$\alpha : \ast \vdash \lambda f : \alpha \to \alpha . \lambda x : \alpha . f (fx) : (\alpha \to \alpha) \to \alpha \to \alpha$}
    \RightLabel{$abst_2$}
    \UnaryInfC{$\emptyset \vdash \lambda \alpha : \ast . \lambda f : \alpha \to \alpha . \lambda x : \alpha . f (fx) : \Pi \alpha : \ast . (\alpha \to \alpha) \to \alpha \to \alpha$}
\end{prooftree}

O resto da Derivação fica como exercício para o leitor

\subsubsection[Propriedades do l2]{Propriedades de $\lambda2$}

A definição de $\alpha$-conversão deve ser acomodada para lidar com tipos $\Pi$:

\begin{definition}[$\alpha$-conversão ou $\alpha$-equivalência]
    \hfil
    \begin{enumerate}
        \item (Renomeando variáveis de termo) \\ $\lambda x : \sigma . M =_{\alpha} \lambda y : \sigma . M^{x \to y}$ se $y \not\in FV(M)$ e $y$ não ocorre como ligante em $M$
        \item (Renomeando variáveis de tipo) \\ $\lambda \alpha : \ast . M =_{\alpha} \lambda \beta : \ast . M[\alpha := \beta]$ se $\beta$ não ocorre em $M$ \\
        $\Pi \alpha : \ast . M =_{\alpha} \Pi \beta : \ast . M[\alpha := \beta]$ se $\beta$ não ocorre em $M$
        \item O resto das definições se segue da definição 1.8
    \end{enumerate}
\end{definition}

Também é possível extender a regra de $\beta$-redução:

\begin{definition} ($\beta$-redução de passo único)
    \hfil
    \begin{enumerate}
        \item (Base, de primeira ordem) $(\lambda : \sigma . M)N \to_{\beta} M[x := N]$
        \item (Base, de segunda ordem) $(\lambda \alpha : \ast . M)T \to_{\beta} M[\alpha := T]$
        \item (Compatibilidade) da mesma forma que definição 1.10
    \end{enumerate}
\end{definition}

Os lemmas definidos no capítulo 2 também podem ser utilizados aqui:

\begin{lemma}
    Os seguintes lemas e teoremas também são válidos para $\lambda2$:
    \begin{itemize}
        \item Lema das variáveis livres
        \item Lema do afinamento
        \item Lema da condensação
        \item Lema da geração
        \item Lema do subtermo
        \item Unicidade dos tipos
        \item Lema da substituição
        \item Teorema de Church-Rosser
        \item Redução do sujeito
        \item Teorema da normalização forte
    \end{itemize}
\end{lemma}

\begin{lemma}[Lema da permutação]
    Se $\Gamma \vdash M : \sigma$ e $\Gamma'$ é uma permutação de $\Gamma$ e um contexto de $\lambda2$ válido, então $\Gamma'$ também é um contexto e $\Gamma' \vdash M : \sigma$.
\end{lemma}

\subsection[O sistema F de Girard]{O Sistema $\mathcal{F}$ de girard}

\end{document}
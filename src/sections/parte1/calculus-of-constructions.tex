\documentclass[../main.tex]{subfiles}

\begin{document}

\section{Calculo de Construções}

\subsection{O Cálculo de Construções e o \texorpdfstring{$\lambda$}{L}-Cubo}

\subsubsection{O Sistema \texorpdfstring{$\lambda C$}{LC}}

O \emph{Calculo de Construções} é a combinação das extenções do ST$\lambda$C vistas anteriormente: O $\lambda 2$, o $\lambda \underline{\omega}$ e o $\lambda P$. Esse sistema foi desenvolvido por Thierry Coquand em 1988. O $C$ vem do seu sobrenome.

A única regra que difere $\lambda P$ de $\lambda C$ é a regra de formação, pois no $\lambda P$, (\emph{form}) se comporta da seguinte forma:

\begin{prooftree}
    \AxiomC{$\Gamma \vdash A : \ast$}
    \AxiomC{$\Gamma, x : A \vdash B : s$}
    \LeftLabel{($form_{\lambda P}$)}
    \BinaryInfC{$\Gamma \vdash \Pi x : A . B : s$}
\end{prooftree}

A primeira premissa garante que $x$ seja um termo, ou seja que $\Pi x : A . B : s$ seja uma relação de tipos dependendo de termos. Porém em $\lambda \underline{\omega}$, tipos podem depender de tipos também, logo $A : \ast$ se torna $A : s$, onde $s$ pode ser tanto $\ast$ ou $\square$. Porém é necessário diferenciar o $s$ da primeira premissa do $s$ da seguinda premissa, logo serão usados os números $1$ e $2$ da seguinte forma:

\begin{prooftree}
    \AxiomC{$\Gamma \vdash A : s_1$}
    \AxiomC{$\Gamma, x : A \vdash B : s_2$}
    \LeftLabel{($form_{\lambda C}$)}
    \BinaryInfC{$\Gamma \vdash \Pi x : A . B : s_2$}
\end{prooftree}

Logo essa regra de formação pode ter quatro formas possíveis a depender da escolha de $s_1$ e $s_2$:

\begin{center}
    \begin{tabular}{|l|l|l|l|l|}
    \hline
    $x : A : s_1$ & $b : B : s_2$ & $(s_1, s_2)$         & $\lambda x : A . b$           & sistema                        \\ \hline
    $\ast$        & $\ast$        & $(\ast, \ast)$       & termos que dependem de termos & $\lambda_{\to}$               \\ \hline
    $\square$     & $\ast$        & $(\square, \ast)$    & termos que dependem de tipos  & $\lambda 2$                   \\ \hline
    $\ast$        & $\square$     & $(\ast, \square)$    & tipos que dependem de termos  & $\lambda P$                   \\ \hline
    $\square$     & $\square$     & $(\square, \square)$ & tipos que dependem de tipos   & $\lambda \underline{\omega}$ \\ \hline
    \end{tabular}
\end{center}

\subsubsection{O \texorpdfstring{$\lambda$}{L}-Cubo}

Todas as extensões do cálculo $\lambda$ simplesmente tipado vistos nos capítulos anteriores se juntam para formar o $\lambda C$, mas são independentes entre si. Cada sistema diferente pode ser unido ao outro formando outro sistema da seguinte forma:

\begin{itemize}
    \item $\lambda \underline{\omega} + \lambda 2 = \lambda \omega$
    \item $\lambda \underline{\omega} + \lambda P = \lambda P \underline{\omega}$
    \item $\lambda P + \lambda 2 = \lambda P 2$
\end{itemize}

Alguns desses sistemas derivados não possuem nomes separados como as extensões iniciais, sendo vistos primariamente como subsistemas de $\lambda C$. O primeiro subsistema dessa lista é, só para relembrar, o Sistema $\mathcal{F} \omega$ de Girard.

Barendregt, no seu estulo desses diversos sistemas e suas ligações, desenvolveu uma forma de perceber esses sistemas em uma espécie de Cubo, onde cada "dimensão" (Largura, altura e profundidade) teria uma característica associada a um sistema diferente da seguinte forma:

\[\begin{tikzcd}
	& {\lambda \omega} && {\lambda C} \\
	\lambda2 && {\lambda P2} \\
	& {\lambda \underline{\omega}} && {\lambda P \underline{\omega}} \\
	{\lambda_{\to}} && {\lambda P}
	\arrow[from=1-2, to=1-4]
	\arrow[from=2-1, to=1-2]
	\arrow[from=2-1, to=2-3]
	\arrow[from=2-3, to=1-4]
	\arrow[from=3-2, to=1-2]
	\arrow[from=3-2, to=3-4]
	\arrow[from=3-4, to=1-4]
	\arrow[from=4-1, to=2-1]
	\arrow[from=4-1, to=3-2]
	\arrow[from=4-1, to=4-3]
	\arrow[from=4-3, to=2-3]
	\arrow[from=4-3, to=3-4]
\end{tikzcd}\]

As regras de inferência do $\lambda C$ são as seguintes:

\begin{itemize}[label=]
    \item (\emph{sort}) $\emptyset \vdash \ast : \square$
    \item \noindent\mbox{%
                \AxiomC{$\Gamma \vdash A : s$}
                \LeftLabel{(\emph{var})}
                \RightLabel{se $x \not\in \Gamma$}
                \UnaryInfC{$\Gamma, x : A \vdash x : A$}
                \DisplayProof
            }
    \item \noindent\mbox{%
                \AxiomC{$\Gamma \vdash A : B$}
                \AxiomC{$\Gamma \vdash C : s$}
                \LeftLabel{(\emph{weak})}
                \RightLabel{se $x \not\in \Gamma$}
                \BinaryInfC{$\Gamma, x : C \vdash A : B$}
                \DisplayProof
            }
    \item \noindent\mbox{%
                \AxiomC{$\Gamma \vdash A : s_1$}
                \AxiomC{$\Gamma, x : A \vdash B : s_2$}
                \LeftLabel{(\emph{form})}
                \BinaryInfC{$\Gamma \vdash \Pi x : A . B : s_2$}
                \DisplayProof
            }
    \item \noindent\mbox{%
            \AxiomC{$\Gamma \vdash M : \Pi x : A . B$}
            \AxiomC{$\Gamma \vdash N : A$}
            \LeftLabel{(\emph{appl})}
            \BinaryInfC{$\Gamma \vdash MN : B[x := N]$}
            \DisplayProof
            }
    \item \noindent\mbox{%
        \AxiomC{$\Gamma, x : A \vdash M : B$}
        \AxiomC{$\Gamma \vdash \Pi x : A . B$}
        \LeftLabel{(\emph{abst})}
        \BinaryInfC{$\Gamma \vdash \lambda x : A . M : \Pi x : A . B$}
        \DisplayProof
    }
    \item \noindent\mbox{%
            \AxiomC{$\Gamma \vdash A : B $}
            \AxiomC{$\Gamma \vdash B' : s$}
            \LeftLabel{(\emph{conv})}
            \RightLabel{se $B =_{\beta} B'$}
            \BinaryInfC{$\Gamma \vdash A : B'$}
            \DisplayProof
    }
\end{itemize}

\subsubsection{Propriedades de \texorpdfstring{$\lambda C$}{LC}}

A maioria das propriedades de $\lambda C$ são extensões das propriedades vistas anteriormente nos seus subsistemas. O que muda de um para o outro é a necessidade de construir definições e proposições que abarquem as diversas dependências que existem no sistema. Muitas das provas também se tornam mais e mais complexas a medida que são adicionadas novas regras e formas de se relacionar os sorts

Primeiro, é necessário definir as expressões de $\lambda C$:

\begin{definition}[Expressões de $\lambda C$, $\mathcal{E}$, \cite{nederpelt2014}]
    O conjunto $\mathcal{E}$ de $\lambda C$-expressões é definido pela seguinte BNF:
    $$\mathcal{E} = V | \square | \ast | (\mathcal{E} \mathcal{E}) | (\lambda V : \mathcal{E} . \mathcal{E}) | \Pi V : \mathcal{E} . \mathcal{E}$$ 
\end{definition}

A notação de parenteses, abstrações sucessivas e aplicações sucessivas é a mesma.

\begin{lemma}[\emph{Lema das Variáveis Livres}, \cite{nederpelt2014}]
    Se $\Gamma \vdash A : B$, então $FV(A)$ e $FV(B) \subseteq dom(\Gamma)$
\end{lemma}

O contexto é dito \emph{bem-formado} se ele forma parte de um juizo derivável:

\begin{definition}[\cite{nederpelt2014}]
    Um contexto $\Gamma$ é \emph{bem formado} se existem $A$ e $B$ tais que $\Gamma \vdash A : B$.
\end{definition}

Com isso, é possível ver os seguintes lemas:

\begin{lemma}[Afinamento, Condensação, Permutação, \cite{nederpelt2014}]
    \hfil
    \begin{enumerate}[label=(\arabic*)]
        \item (\emph{Afinamento}) Sejam $\Gamma '$ e $\Gamma ''$ contextos tais que $\Gamma' \subseteq \Gamma ''$. Se $\Gamma ' \vdash A : B$ e $\Gamma ''$ é bem formado, então $\Gamma '' \vdash A : B$
        \item (\emph{Permutação}) Sejam $\Gamma '$ e $\Gamma ''$ dois contextos e seja $\Gamma ''$ a permutação de $\Gamma '$. Se $\Gamma ' \vdash A : B$ e $\Gamma ''$ é bem formado, então $\Gamma '' \vdash A : B$.
        \item (\emph{Condensação}) Se $\Gamma ', x : A, \Gamma '' \vdash B : C$ e $x$ não ocorre em $\Gamma ''$, $B$ ou $C$, então $\Gamma', \Gamma'' \vdash B : C$.
    \end{enumerate}
\end{lemma}

Também é interessante saber se uma expressão é \emph{legal} ou não:

\begin{definition}[\cite{nederpelt2014}]
    Uma expressão $M$ em $\lambda C$ é \emph{legal} se existe $\Gamma$ e $N$ tais que $\Gamma \vdash M : N$ ou $\Gamma \vdash N : M$ (Ou seja, quando $M$ é \emph{tipável} ou \emph{habitável})
\end{definition}

\begin{lemma}[Lema da subexpressão, \cite{nederpelt2014}]
    Se $M$ é legal, então toda subexpressão de $M$ é legal
\end{lemma}

Outro lema importante:

\begin{lemma}[Lema da unicidade de tipos a menos que conversão, \cite{nederpelt2014}]
    Se $\Gamma \vdash A : B_1$ e $\Gamma \vdash A : B_2$, então $B_1 =_{\beta} B_2$
\end{lemma}

\begin{lemma}[Lema da substituição, \cite{nederpelt2014}]
    Seja $\Gamma ', x : A, \Gamma '' \vdash B : C$ e $\Gamma ' \vdash D : A$, então $\Gamma ', \Gamma '' [x := D] \vdash B[x := D] : C [x := D]$
\end{lemma}

Esse lema fala que é possível substituir $D$ em $\Gamma ', x : A, \Gamma '' \vdash B : C$ caso $D$ e $x : A$ possuam o mesmo tipo. Como tipos também podem possuir termos, essa substituição vai ocorer em todos os lugares.

Um teorema presente também em $\lambda C$ é o Teorema de Church-Rosser:

\begin{theorem}[Teorema de Church-Rosser, \cite{nederpelt2014}]
    A propriedade de Church-Rosser é válida para $\lambda C$, ou seja, se $M$ está em $\mathcal{E}$, $M \twoheadrightarrow_{\beta} N_1$ e $M \twoheadrightarrow_{\beta} N_2$, então existe um $N_3$ tal que $N_1 \twoheadrightarrow_{\beta} N_3$ e $N_2 \twoheadrightarrow_{\beta} N_3$
\end{theorem}

\begin{corollary}[\cite{nederpelt2014}]
    Suponha que $M$ e $N$ estão em $\mathcal{E}$ e $M =_{\beta} N$, então existe um $L$ tal que $M \twoheadrightarrow_{\beta} L$ e $N \twoheadrightarrow_{\beta} L$
\end{corollary}

\begin{lemma}[Redução de sujeito, \cite{nederpelt2014}]
    Se $\Gamma \vdash A : B$ e $A \twoheadrightarrow_{\beta} A'$, então $\Gamma \vdash A' : B$
\end{lemma}

Outro teorema muito importante que está presente em $\lambda C$ é o da normalização forte:

\begin{theorem}[Teorema da Normalização Forte, \cite{nederpelt2014}]
    Todo termo legal $M$ é fortemente normalizável
\end{theorem}

Finalmente, de um ponto de vista computacional, a questão que fica é saber se a Boa-tipagem e a checagem de tipos é decidível ou não em $\lambda C$, e a resposta é sim:

\begin{theorem}[Decidabilidade em $\lambda C$, \cite{nederpelt2014}]
    Em $\lambda C$ e seus subsistemas, as questões de Boa-tipagem e Checagem de tipos são decidíveis
\end{theorem}

Logo, é possível desenvolver um \emph{programa de computador} que resolve esses problemas automaticamente. Um desses programas é o \emph{Provador de Teoremas} \textbf{Coq}.

A questão de Encontrar os termos só é decidível em $\lambda_{\to}$ e $\lambda \underline{\omega}$, mas indecidível nos outros sistemas.

\subsection{A lógica no \texorpdfstring{$\lambda$C}{LC}}

No capítulo anterior, sobre o sistema $\lambda P$, foi codificada uma lógica na teoria dos tipos que possuia como operadores a \emph{implicação} e o \emph{quantificador universal}. Como $\lambda P$ é parte de $\lambda C$, essa codificação pode ser mantida sem problemas.

Para se ter uma lógica proposicional além de \emph{minima}, é necessário introduzir outros conectivos lógicos como a negação ($\neg$), a conjunção ($\wedge$) e a disjunção ($\vee$). Isso não pode ser feito em $\lambda P$ mas pode ser feito em $\lambda C$

\subsubsection{Absurdo e negação na teoria dos tipos}

Começando com a negação. É natural de se pensar a negação $\neg A$ como uma implicação $A \Rightarrow \bot$ que leva de $A$ para o "absurdo", chamado de \emph{contradição}. Logo $\neg A$ é lido como "$A$ implica no absurdo".

Para isso, se torna necessário codificar o absurdo, que será codificado na seguinte forma: $$\bot \equiv \Pi \alpha : \ast . \alpha$$. Fica a cargo do leitor mostrar que esse tipo é válido em $\lambda C$. 

Uma característica do absurdo é a chamada \emph{regra da explosão}, ou \emph{ex falso quodlibet} (do falso se segue qualquer coisa):

\emph{Se $\bot$ é verdadeiro, então toda proposição é verdadeira}

No sentido da teoria dos tipos, é a mesma coisa que dizer que:

\emph{Se $\bot$ é habitado, qualquer proposição é habitada}

Na interpretação construtiva, pode-se construir uma função:

"Se $M$ é habitante de $\bot$, então existe uma função que mapeia uma proposição arbitrária $\alpha$ em um habitante desse mesmo $\alpha$"

Essa função possui o tipo $\Pi \alpha : \ast . \alpha$ e se $f$ possui esse tipo, entao, pela aplicação, $fA : \alpha[\alpha := A] \equiv A$. Ou seja, $fA$ habita $A$ para qualquer tipo $A$ Então:

"Se $M$ for um habitante de $\bot$, então existe uma função $f$ que habita $\Pi \alpha : \ast . \alpha$"

De forma contrária, tendo $f$, então todas as proposições são verdadeiras, o que é absurdo, logo existe o absurdo.

Em suma: $\bot$ é habitado, se e somente se, $\Pi \alpha : \ast . \alpha$ for habitado.

A regra da eliminação de $\bot$ é feita da seguinte forma:

\begin{prooftree}
    \def\fCenter{\mbox{\ $\vdash$\ }}
    \AxiomC{$\bot$}
    \LeftLabel{($\bot$ - \emph{elim})}
    \UnaryInfC{$A$}
\end{prooftree}

Na dedução natural, essa regra não aparece com esse nome específico, mas sim relacionado à negação.

Para terminar, é necessário discutir a qual subsistema de $\lambda C$ $\bot$ pertence. No tipo $\Pi \alpha : \ast . \alpha$, como $\alpha : \ast : \square$, $s_1 = \square$ e $s_2 = \ast$.

Tendo declarado o absurdo, se torna fácil definir a negação. $\neg A \equiv A \to \bot$ é abreviação de $\Pi x : A . \bot$. Como $A : \ast$ e $\bot : \ast$, então $(s_1, s_2) = (\ast, \ast)$. Como trata-se de um tipo que usa $\bot$, então seu subsistema mínimo é $\lambda 2$.

É possível definir a regra da introdução do absurdo da seguinte forma:

\begin{prooftree}
    \def\fCenter{\mbox{\ $\vdash$\ }}
    \AxiomC{$A$}
    \AxiomC{$\neg A$}
    \LeftLabel{($\bot$ - \emph{intro})}
    \BinaryInfC{$\bot$}
\end{prooftree}

mas como $\neg A \equiv A \to \bot$, isso é o mesmo que:

\begin{prooftree}
    \def\fCenter{\mbox{\ $\vdash$\ }}
    \AxiomC{$A$}
    \AxiomC{$A \to \bot$}
    \LeftLabel{($\bot$ - \emph{intro})}
    \BinaryInfC{$\bot$}
\end{prooftree}

que é nada mais que uma repaginação da regra ($\Rightarrow$-\emph{elim}). Da mesma forma, as regras de introdução e eliminação da negação também são repaginações das regras de introdução e eliminação da implicação:

\begin{center}
    \AxiomC{$A$}
    \AxiomC{$A \to \bot$}
    \LeftLabel{($\neg$ - \emph{elim})}
    \RightLabel{$\qquad$}
    \BinaryInfC{$\bot$}
    \DisplayProof
    \AxiomC{$A$}
    \AxiomC{$A \Rightarrow B$}
    \LeftLabel{($\Rightarrow$ - \emph{elim})}
    \BinaryInfC{$B$}
    \DisplayProof

    \hfil

    \AxiomC{$[A]$}
    \noLine
    \UnaryInfC{$\vdots$}
    \noLine
    \UnaryInfC{$\bot$}
    \LeftLabel{($\neg$-\emph{intro})}
    \UnaryInfC{$\neg A$}
    \DisplayProof
    \AxiomC{$[A]$}
    \noLine
    \UnaryInfC{$\vdots$}
    \noLine
    \UnaryInfC{$B$}
    \LeftLabel{$\qquad$($\neg$-\emph{intro})}
    \UnaryInfC{$A \to B$}
    \DisplayProof
\end{center}

\subsubsection{Conjunção e Disjunção na teoria dos tipos}

\emph{I. Conjunção}

A conjunção $A \land B$ é verdadeira se, e somente se, \emph{ambos} $A$ e $B$ são verdadeiros. Existe uma codificação da conjunção em $\lambda 2$ que é:

$$A \land B \equiv \Pi C : \ast . (A \to B \to C) \to C$$

chamada de \emph{codificação de segunda ordem} da conjunção, sendo mais geral que a codificação \emph{de primeira ordem}:

$$A \land B \equiv (A \to \neq B)$$

A codificação de segunda ordem é mais geral pois a codificação de primeira ordem só funciona na lógica clássica.

Mas o que quer dizer de fato essa codificação? Lendo o $\Pi$ como um quantificador universal, pode-se ler como: "para todo $C$, ($A$ implica em $B$ implica em $C$) implica em $C$". Ou: "Se $A$ e $B$ juntos implicam em $C$, então $C$ é válido por si só".

Essa codificação é chamada de \emph{segunda ordem} pois ela é generalizada sobre todas as proposições $C : \ast$ que são objetos de segunda ordem.

As regras de inferência de $\land$ na dedução natural e na teoria dos tipos são as seguintes:

\begin{center}
    \AxiomC{$A$}
    \AxiomC{$B$}
    \LeftLabel{($\land$-\emph{intro})}
    \BinaryInfC{$A \land B$}
    \DisplayProof
    \AxiomC{$A$}
    \AxiomC{$B$}
    \LeftLabel{$\qquad$($\land$-\emph{intro})}
    \BinaryInfC{$\Pi C : \ast . (A \to B \to C) \to C$}
    \DisplayProof
    
    \hfill

    \AxiomC{$A \land B$}
    \LeftLabel{($\land$-\emph{elim-left})}
    \UnaryInfC{$A$}
    \DisplayProof
    \AxiomC{$\Pi C : \ast . (A \to B \to C) \to C$}
    \LeftLabel{$\qquad$($\land$-\emph{intro-left})}
    \UnaryInfC{$A$}
    \DisplayProof   

    \hfill
    
    \AxiomC{$A \land B$}
    \LeftLabel{($\land$-\emph{elim-right})}
    \UnaryInfC{$B$}
    \DisplayProof
    \AxiomC{$\Pi C : \ast . (A \to B \to C) \to C$}
    \LeftLabel{$\qquad$($\land$-\emph{intro-right})}
    \UnaryInfC{$B$}
    \DisplayProof 
\end{center}

Para provar que essas regras são válidas em $\lambda C$, é necessário encontrar um termo que obedeça a essas regras da seguinte forma:

\begin{center}
    \AxiomC{$\Gamma \vdash a : A$}
    \AxiomC{$\Gamma \vdash b : B$}
    \LeftLabel{$\qquad$($\land$-\emph{intro-sec-tt})}
    \BinaryInfC{$\Gamma \vdash ?_1 : \Pi C : \ast . (A \to B \to C) \to C$}
    \DisplayProof

    \AxiomC{$\Gamma \vdash c : \Pi C : \ast . (A \to B \to C) \to C$}
    \LeftLabel{$\qquad$($\land$-\emph{intro-sec-left-tt})}
    \UnaryInfC{$\Gamma \vdash ?_2 : A$}
    \DisplayProof   

    \AxiomC{$\Gamma \vdash c : \Pi C : \ast . (A \to B \to C) \to C$}
    \LeftLabel{$\qquad$($\land$-\emph{intro-sec-right-tt})}
    \UnaryInfC{$\Gamma \vdash ?_3 : B$}
    \DisplayProof   
\end{center}

Como exemplo, segue a derivação da regra de introdução:

\begin{center}
    \AxiomC{$\Gamma \vdash a : A$}
    \AxiomC{$\Gamma \vdash b : B$}
    \BinaryInfC{$\dots$}
    \LeftLabel{$?$}
    \UnaryInfC{$\Gamma \vdash ?_1 : \Pi C : \ast . (A \to B \to C) \to C$}
    \DisplayProof
\end{center}

\begin{center}
    \AxiomC{$\Gamma, C : \ast, z : (A \to B \to C) \vdash b : B$}
    \AxiomC{$\Gamma, C : \ast, z : (A \to B \to C) \vdash a : A$}
    \AxiomC{$\Gamma, C : \ast, z : (A \to B \to C) \vdash z : (A \to B \to C)$}
    \LeftLabel{(\emph{appl})}
    \BinaryInfC{$\Gamma, C : \ast, z : (A \to B \to C) \vdash za : B \to C$}
    \LeftLabel{(\emph{appl})}
    \BinaryInfC{$\Gamma, C : \ast, z : (A \to B \to C) \vdash zab : C$}
    \LeftLabel{(\emph{abst})}
    \UnaryInfC{$\Gamma, C : \ast \vdash \lambda z : (A \to B \to C) . zab : (A \to B \to C) \to C$}
    \LeftLabel{(\emph{abst})}
    \UnaryInfC{$\Gamma \vdash \lambda C : \ast . \lambda z : (A \to B \to C) . zab : \Pi C : \ast . (A \to B \to C) \to C$}
    \DisplayProof
\end{center}

Com $\Gamma \equiv A : \ast, B : \ast, a : A, b : B$



\emph{II. Disjunção}

Para a disjunção também existe uma codificação de primeira e segunda ordem. A codificação de segunda ordem é:

$$A \lor B \equiv \Pi C : \ast . (A \to C) \to (B \to C) \to C$$

(A codificação de primeira ordem é $A \lor B \equiv \neg A \to B$, mas assim como na conjunção, ela só serve para a lógica clássica)

Assim como na conjunção, essa codificação possui uma interpretação literal como:

\emph{Para todo $C$, ($A \to C$ implica que ($B \to C$ implica em $C$))}

Isso é o mesmo que dizer que:

"Se $A$ implica em $C$ e $B$ implica em $C$, então $C$ se segue"

Ou

"Se $A$ ou $B$ implica em em $C$, então $C$ se segue"

As regras de inferência de $\lor$ na dedução natural e na teoria dos tipos são as seguintes:

\begin{center}
    \AxiomC{$A$}
    \LeftLabel{($\lor$-\emph{intro-left})}
    \UnaryInfC{$A \lor B$}
    \DisplayProof
    \AxiomC{$A$}
    \LeftLabel{$\qquad$($\lor$-\emph{intro-left})}
    \UnaryInfC{$\Pi C : \ast . (A \to C) \to (B \to C) \to C$}
    \DisplayProof
    
    \hfill

    \AxiomC{$B$}
    \LeftLabel{($\lor$-\emph{intro-right})}
    \UnaryInfC{$A \lor B$}
    \DisplayProof
    \AxiomC{$B$}
    \LeftLabel{$\qquad$($\lor$-\emph{intro-right})}
    \UnaryInfC{$\Pi C : \ast . (A \to C) \to (B \to C) \to C$}
    \DisplayProof

    \hfill
    
    \AxiomC{$A \lor B$}
    \AxiomC{$A \Rightarrow C$}
    \AxiomC{$B \Rightarrow C$}
    \LeftLabel{($\lor$-\emph{elim})}
    \TrinaryInfC{$B$}
    \DisplayProof
    \AxiomC{$\Pi D : \ast . (A \to D) \to (B \to D) \to D$}
    \AxiomC{$A \to C$}
    \AxiomC{$B \to C$}
    \LeftLabel{$\qquad$($\lor$-\emph{elim})}
    \TrinaryInfC{$C$}
    \DisplayProof 
\end{center}

Para as regras da dedução natural:

As regras (\emph{intro}) são o mesmo que: se $A$ é verdadeira por si só, então $A \lor B$ também será. O mesmo para $B$.

A regra (\emph{elim}) descreve bem o comportamento dito acima da implicação

A derivação da (\emph{elim}) pode ser feita seguindo o seguinte termo: $$\lambda x : \Pi D : \ast . (A \to D) \to (B \to D) \to D . \lambda y : A \to C . \lambda z : B \to C . xCyz : C$$

A árvore de derivação fica para ser construida pelo leitor.

Uma nota importante é que os operadores lógicos vistos até então podem ser construidos quantificando em cima de qualquer proposição(-como-tipo), nesse caso eles se tornam:

\begin{align*}
    \neg &\equiv \lambda \alpha : \ast . (\alpha \to \bot) \\
    \land &\equiv \lambda \alpha : \ast . \lambda \beta : \ast . \Pi \gamma : \ast (\alpha \to \beta \to \gamma) \to \gamma \\
    \lor &\equiv \lambda \alpha : \ast . \lambda \beta : \ast . \Pi \gamma : \ast (\alpha \to \gamma) \to (\beta \to \gamma) \to \gamma 
\end{align*}

Exemplo:

\begin{align*}
    A \land B &\equiv (\lambda \alpha : \ast . \lambda \beta : \ast . \Pi \gamma : \ast (\alpha \to \beta \to \gamma) \to \gamma)AB \\
              &\to_{\beta} (\lambda \beta : \ast . \Pi \gamma : \ast (A \to \beta \to \gamma) \to \gamma)B \\
              &\to_{\beta} \Pi \gamma : \ast (A \to B \to \gamma) \to \gamma \\
              &\equiv \Pi C : \ast (A \to B \to C) \to C \\
\end{align*}

Enquanto os outros operadores menos gerais estão em $\lambda 2$, esses mais gerais estão no sistema $\lambda \omega$

\emph{III. Biimplicação}

A Biimplicação, ou também "Se, e somente se", na lógica é o operador $\Leftrightarrow$ que não possui uma tipagem direta em $\lambda C$, mas pode ser construido usando a seguinte equivalência:

$$A \Leftrightarrow B \equiv (A \Rightarrow B) \land (B \Rightarrow A)$$

\subsubsection{Exemplo de Derivação}

Para mostrar que a lógica funciona na teoria dos tipos, é importante demonstrar um exemplo de tautologia, como a seguinte:

$$(A \lor B) \Rightarrow (\neg A \Rightarrow B)$$

Como cada operador desse possui uma codificação na teoria dos tipos, essa tautologia pode ser reescrita na forma:


$$(\Pi C : \ast . (A \to C) \to (B \to C) \to C) \to (A \to \bot) \to B$$

Como $A$ e $B$ são proposições arbitrárias, elas ficam no contexto da derivação.

Para dar um norte, a derivação na dedução natural dessa tautologia é a seguinte:

\begin{prooftree}
    \AxiomC{$[A \lor B]^1$}
    \AxiomC{$[A]^3$}
    \AxiomC{$[\neg A]^2$}
    \LeftLabel{($\neg$-\emph{elim})}
    \BinaryInfC{$\bot$}
    \LeftLabel{($\bot$-\emph{elim})}
    \UnaryInfC{$B$}
    \AxiomC{$[B]^3$}
    \LeftLabel{($\lor$-\emph{elim})}
    \RightLabel{$3$}
    \TrinaryInfC{$B$}
    \LeftLabel{($\Rightarrow$-\emph{intro})}
    \RightLabel{$2$}
    \UnaryInfC{$\neg A \Rightarrow B$}
    \LeftLabel{($\Rightarrow$-\emph{intro})}
    \RightLabel{$1$}
    \UnaryInfC{$(A \lor B) \Rightarrow (\neg A \Rightarrow B)$}
\end{prooftree}

Essa árvore pode dar uma noção de por onde começar

Primeiro, é necessário um termo $c : (\Pi C : \ast . (A \to C) \to (B \to C) \to C)$ para ser o $A \lor B$. Também de um termo $n : A \to \bot$ para ser o $\neg A$.

Mas a dúvida então fica em como encontrar o $B$. para isso fica claro pela dedução natural que o próximo passo seria eliminar o $\lor$ usando $B$ no lugar de $C$, logo seria necessário o tipo: $(A \to B) \to (B \to B) \to B$, gerado por $cB$. Dado esse tipo, só se torna necessário encontrar dois termos com tipos $A \to B$ e $B \to B$. O segundo é um termo identidade $\lambda v : B . v$ e o primeiro tem que ser um termo $\lambda u : A . M : A \to B$. 

Para gerar o termo $M$ é necessário lembrar que o tipo absurdo $\bot$ é o mesmo que $\Pi \alpha : \ast . \alpha$ e que $(\Pi \alpha : \ast . \alpha)B \to_{\beta} B$. Logo, $M \equiv uvB$.

Fica-se então com o seguinte termo:

\begin{align*}
    \lambda c : (\Pi C : \ast . (A \to C) \to (B \to C) \to C) . \lambda n : (A \to \bot) . cB(\lambda u : A . nuB)(\lambda v : B . v) \\  : (\Pi C : \ast . (A \to C) \to (B \to C) \to C) \to (A \to \bot) \to B
\end{align*}


A árvore de derivação completa (retirando a segunda premissa da abstração e fins que não são fins de verdade) está na próxima página:

\newpage

\begin{center}
    \rotatebox{90}{
        \AxiomC{$\Gamma \vdash c: \Pi C : \ast . (A \to C) \to (B \to C) \to C$}
        \AxiomC{$\Gamma \vdash B: \ast$}
        \LeftLabel{(\emph{appl})}
        \BinaryInfC{$\Gamma \vdash cB: (A \to B) \to (B \to B) \to B$}


        \AxiomC{$\Gamma, u : A \vdash n : A \to \bot$}
        \AxiomC{$\Gamma, u : A \vdash u : A$}
        \LeftLabel{(\emph{appl})}
        \BinaryInfC{$\Gamma, u : A \vdash nu : \bot \equiv \Pi \alpha : \ast . \alpha$}
        \AxiomC{$\Gamma, u : A \vdash B : \ast$}
        \LeftLabel{(\emph{appl})}
        \BinaryInfC{$\Gamma, u : A \vdash nuB : B$}
        \LeftLabel{(\emph{abst})}
        \UnaryInfC{$\Gamma \vdash \lambda u : A . nuB : A \to B$}
        \LeftLabel{(\emph{appl})}
        \BinaryInfC{$\Gamma \vdash cB(\lambda u : A . nuB) : (B \to B) \to B$}
        \AxiomC{$\Gamma, v : B \vdash v : B$ }
        \LeftLabel{(\emph{abst})}
    \UnaryInfC{$\Gamma \vdash \lambda v : B . v : (B \to B)$}
        \LeftLabel{(\emph{appl})}
        \BinaryInfC{$A : \ast, B : \ast, c : (\Pi C : \ast . (A \to C) \to (B \to C) \to C), n (A \to \bot) \vdash cB(\lambda u : A . nuB)(\lambda v : B . v) : B$}
        \LeftLabel{(\emph{abst})}
        \UnaryInfC{$A : \ast, B : \ast, c : (\Pi C : \ast . (A \to C) \to (B \to C) \to C) \vdash \lambda n : (A \to \bot) . . cB(\lambda u : A . nuB)(\lambda v : B . v) : (A \to \bot) \to B$}
        \LeftLabel{(\emph{abst})}
        \UnaryInfC{\stackanchor{$A : \ast, B : \ast \vdash \lambda c : (\Pi C : \ast . (A \to C) \to (B \to C) \to C) . \lambda n : (A \to \bot) . cB(\lambda u : A . nuB)(\lambda v : B . v) :$}{$(\Pi C : \ast . (A \to C) \to (B \to C) \to C) \to (A \to \bot) \to B$}}
        \DisplayProof
    }

\end{center}

\newpage

\subsubsection{Lógica clássica em \texorpdfstring{$\lambda$C}{LC}}

A lógica vista até então não é \emph{clássica}, mas sim \emph{intuicionista} ou \emph{construtivista}. Essa lógica é um pouco mais fraca que a lógica clássica, pois ela não possui a \emph{lei do terceiro excluido} (LEM) que diz que $A \lor \neg A$ é válido para qualquer $A$. Ela também não possui a lei da \emph{dupla negação} (DN) que diz que $\neq \neq A \Rightarrow A$ é válido para qualquer $A$.

Para obter essa lógica clássica é necessário fazer uma extensão da lógica vista até então. Para isso, não é necessário introduzir ambas LEM e DN, pois uma implica na outra, dessa forma só será introduzida a LEM. 

Mas como introduzir uma regra que não é derivável em $\lambda C$? para isso, é necessário assumi-la como \emph{suposição}, também chamado de \emph{postulado} ou \emph{axioma}, no contexto. A LEM será introduzida da seguinte forma:

$$i_{LEM} = \Pi \alpha : \ast . \alpha \lor \neg \alpha$$

Um teste inicial interessante é provar o que foi dito anteriormente que $\lambda C + i_{LEM}$ prova \emph{DN}.

\begin{prooftree}
    
    \AxiomC{$?$}
    \LeftLabel{$?$}
    \UnaryInfC{$ i_{LEM} \vdash ? : \Pi \beta : \ast . \neg \neg \beta \to \beta$}
\end{prooftree}

\begin{prooftree}
    \AxiomC{$?$}
    \LeftLabel{$?$}
    \UnaryInfC{$ i_{LEM}, \beta : \ast \vdash ? : \neg \neg \beta \to \beta$}
    \LeftLabel{(\emph{abst})}
    \UnaryInfC{$ i_{LEM} \vdash \lambda \beta : \ast . ? : \Pi \beta : \ast . \neg \neg \beta \to \beta$}
\end{prooftree}

\begin{prooftree}
    \AxiomC{$?$}
    \LeftLabel{$?$}
    \UnaryInfC{$ i_{LEM}, \beta : \ast, x : \neg \neg \beta \vdash ? : \beta$}
    \LeftLabel{(\emph{abst})}
    \UnaryInfC{$ i_{LEM}, \beta : \ast \vdash \lambda x : \neg \neg \beta . ? : \neg \neg \beta \to \beta$}
    \LeftLabel{(\emph{abst})}
    \UnaryInfC{$ i_{LEM} \vdash \lambda \beta : \ast . \lambda x : \neg \neg \beta . ? : \Pi \beta : \ast . \neg \neg \beta \to \beta$}
\end{prooftree}

Nessa etapa, é necessário descobrir um termo $M : \beta$. Mas tendo em vista que $i_{LEM}\beta \equiv \beta \lor \neg \beta \equiv \Pi \gamma : \ast . (\beta \to \gamma) \to (\neg \beta \to \gamma) \to \gamma$, é possível construir o tipo $i_{LEM}\beta \equiv\beta \equiv (\beta \to \beta) \to (\neg \beta \to \beta) \to \beta$.

A primeira entrada desse tipo pode ser construida usando um termo que é uma identidade $\lambda u : \beta . u$, já o segundo pode ser construido pensando que $\neg \neg \beta \equiv (\neg \beta \to \bot)$. Sendo $x : \neg \neg \beta$, ao aplicar esse termo a um termo $z : \neg \beta$, tem-se $xz : \bot$, mas pela definição de $\bot$, $\bot\beta \to_{\beta} \beta$. logo $xz\beta : \beta$. Então o segundo termo é: $\lambda z : \neg \beta . xz\beta : \neg \beta \to \beta$.

O termo $M$ então se torna $M \equiv i_{LEM}\beta\beta(\lambda y : \beta . y)(\lambda z : \neg \beta . xz\beta) : \beta$.

A árvore final se torna:

Obs: árvore incompleta; $\Gamma \equiv i_{LEM}, \beta : \ast, x : \neg \neg \beta$

\newpage
\begin{center}
    \rotatebox{90}{
        \AxiomC{$\vdots$}
        \LeftLabel{(\emph{appl})}
        \UnaryInfC{$\Gamma \vdash i_{LEM}\beta\beta : (\beta \to \beta) \to (\neg \beta \to \beta) \to \beta$}

        \AxiomC{$\Gamma, y : \beta \vdash y : \beta \to \beta$}
        \LeftLabel{(\emph{abst})}
        \UnaryInfC{$\Gamma \vdash \lambda y : \beta . y : \beta \to \beta$}
        \LeftLabel{(\emph{appl})}
        \BinaryInfC{$\Gamma \vdash i_{LEM}\beta\beta(\lambda y : \beta . y) : (\neg \beta \to \beta) \to \beta$}

        \AxiomC{$\Gamma, z : \neg \beta  \vdash x : \neg \neg \beta$}
        \AxiomC{$\Gamma, z : \neg \beta  \vdash z : \neg \beta$}
        \LeftLabel{(\emph{appl})}
        \BinaryInfC{$\Gamma, z : \neg \beta  \vdash xz : \bot \equiv \Pi \alpha : \ast . \alpha$}
        \AxiomC{$\Gamma, z : \neg \beta  \vdash \beta : \ast$}
        \LeftLabel{(\emph{appl})}
        \BinaryInfC{$\Gamma, z : \neg \beta  \vdash xz\beta : \beta$}
        \LeftLabel{(\emph{abst})}
        \UnaryInfC{$\Gamma \vdash \lambda z : \neg \beta . xz\beta : \beta$}
        \LeftLabel{(\emph{appl})}
        \BinaryInfC{$ i_{LEM}, \beta : \ast, x : \neg \neg \beta \vdash i_{LEM}\beta\beta(\lambda y : \beta . y)(\lambda z : \neg \beta . xz\beta) : \beta$}
        \LeftLabel{(\emph{appl})}
        \UnaryInfC{$ i_{LEM}, \beta : \ast, x : \neg \neg \beta \vdash i_{LEM}\beta\beta(\lambda y : \beta . y)(\lambda z : \neg \beta . xz\beta) : \beta$}
        \LeftLabel{(\emph{abst})}
        \UnaryInfC{$ i_{LEM}, \beta : \ast \vdash \lambda x : \neg \neg \beta . i_{LEM}\beta\beta(\lambda y : \beta . y)(\lambda z : \neg \beta . xz\beta) : \neg \neg \beta \to \beta$}
        \LeftLabel{(\emph{abst})}
        \UnaryInfC{$ i_{LEM} \vdash \lambda \beta : \ast . \lambda x : \neg \neg \beta . i_{LEM}\beta\beta(\lambda y : \beta . y)(\lambda z : \neg \beta . xz\beta) : \Pi \beta : \ast . \neg \neg \beta \to \beta$}
        \DisplayProof
    }

\end{center}

\newpage

\subsubsection{Lógica de predicados em \texorpdfstring{$\lambda$C}{LC}}




\end{document}
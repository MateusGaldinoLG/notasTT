\documentclass[../main.tex]{subfiles}

\begin{document}

\section{Calculo de Construções}

\subsection{O Cálculo de Construções e o \texorpdfstring{$\lambda$}{L}-Cubo}

\subsubsection{O Sistema \texorpdfstring{$\lambda C$}{LC}}

O \emph{Calculo de Construções} é a combinação das extenções do ST$\lambda$C vistas anteriormente: O $\lambda 2$, o $\lambda \underline{\omega}$ e o $\lambda P$. Esse sistema foi desenvolvido por Thierry Coquand em 1988. O $C$ vem do seu sobrenome.

A única regra que difere $\lambda P$ de $\lambda C$ é a regra de formação, pois no $\lambda P$, (\emph{form}) se comporta da seguinte forma:

\begin{prooftree}
    \AxiomC{$\Gamma \vdash A : \ast$}
    \AxiomC{$\Gamma, x : A \vdash B : s$}
    \LeftLabel{($form_{\lambda P}$)}
    \BinaryInfC{$\Gamma \vdash \Pi x : A . B : s$}
\end{prooftree}

A primeira premissa garante que $x$ seja um termo, ou seja que $\Pi x : A . B : s$ seja uma relação de tipos dependendo de termos. Porém em $\lambda \underline{\omega}$, tipos podem depender de tipos também, logo $A : \ast$ se torna $A : s$, onde $s$ pode ser tanto $\ast$ ou $\square$. Porém é necessário diferenciar o $s$ da primeira premissa do $s$ da seguinda premissa, logo serão usados os números $1$ e $2$ da seguinte forma:

\begin{prooftree}
    \AxiomC{$\Gamma \vdash A : s_1$}
    \AxiomC{$\Gamma, x : A \vdash B : s_2$}
    \LeftLabel{($form_{\lambda C}$)}
    \BinaryInfC{$\Gamma \vdash \Pi x : A . B : s_2$}
\end{prooftree}

Logo essa regra de formação pode ter quatro formas possíveis a depender da escolha de $s_1$ e $s_2$:

\begin{center}[]
    \begin{tabular}{|l|l|l|l|l|}
    \hline
    $x : A : s_1$ & $b : B : s_2$ & $(s_1, s_2)$         & $\lambda x : A . b$           & sistema                        \\ \hline
    $\ast$        & $\ast$        & $(\ast, \ast)$       & termos que dependem de termos & $\lambda_{\to}$               \\ \hline
    $\square$     & $\ast$        & $(\square, \ast)$    & termos que dependem de tipos  & $\lambda 2$                   \\ \hline
    $\ast$        & $\square$     & $(\ast, \square)$    & tipos que dependem de termos  & $\lambda P$                   \\ \hline
    $\square$     & $\square$     & $(\square, \square)$ & tipos que dependem de tipos   & $\lambda \underline{\omega}$ \\ \hline
    \end{tabular}
\end{center}

\subsubsection{O \texorpdfstring{$\lambda$}{L}-Cubo}

Todas as extensões do cálculo $\lambda$ simplesmente tipado vistos nos capítulos anteriores se juntam para formar o $\lambda C$, mas são independentes entre si. Cada sistema diferente pode ser unido ao outro formando outro sistema da seguinte forma:

\begin{itemize}
    \item $\lambda \underline{\omega} + \lambda 2 = \lambda \omega$
    \item $\lambda \underline{\omega} + \lambda P = \lambda P \underline{\omega}$
    \item $\lambda P + \lambda 2 = \lambda P 2$
\end{itemize}

Alguns desses sistemas derivados não possuem nomes separados como as extensões iniciais, sendo vistos primariamente como subsistemas de $\lambda C$. O primeiro subsistema dessa lista é, só para relembrar, o Sistema $\mathcal{F} \omega$ de Girard.

Barendregt, no seu estulo desses diversos sistemas e suas ligações, desenvolveu uma forma de perceber esses sistemas em uma espécie de Cubo, onde cada "dimensão" (Largura, altura e profundidade) teria uma característica associada a um sistema diferente da seguinte forma:

\[\begin{tikzcd}
	& {\lambda \omega} && {\lambda C} \\
	\lambda2 && {\lambda P2} \\
	& {\lambda \underline{\omega}} && {\lambda P \underline{\omega}} \\
	{\lambda_{\to}} && {\lambda P}
	\arrow[from=1-2, to=1-4]
	\arrow[from=2-1, to=1-2]
	\arrow[from=2-1, to=2-3]
	\arrow[from=2-3, to=1-4]
	\arrow[from=3-2, to=1-2]
	\arrow[from=3-2, to=3-4]
	\arrow[from=3-4, to=1-4]
	\arrow[from=4-1, to=2-1]
	\arrow[from=4-1, to=3-2]
	\arrow[from=4-1, to=4-3]
	\arrow[from=4-3, to=2-3]
	\arrow[from=4-3, to=3-4]
\end{tikzcd}\]

As regras de inferência do $\lambda C$ são as seguintes:

\begin{itemize}[label=]
    \item (\emph{sort}) $\emptyset \vdash \ast : \square$
    \item \noindent\mbox{%
                \AxiomC{$\Gamma \vdash A : s$}
                \LeftLabel{(\emph{var})}
                \RightLabel{se $x \not\in \Gamma$}
                \UnaryInfC{$\Gamma, x : A \vdash x : A$}
                \DisplayProof
            }
    \item \noindent\mbox{%
                \AxiomC{$\Gamma \vdash A : B$}
                \AxiomC{$\Gamma \vdash C : s$}
                \LeftLabel{(\emph{weak})}
                \RightLabel{se $x \not\in \Gamma$}
                \BinaryInfC{$\Gamma, x : C \vdash A : B$}
                \DisplayProof
            }
    \item \noindent\mbox{%
                \AxiomC{$\Gamma \vdash A : s_1$}
                \AxiomC{$\Gamma, x : A \vdash B : s_2$}
                \LeftLabel{(\emph{form})}
                \BinaryInfC{$\Gamma \vdash \Pi x : A . B : s_2$}
                \DisplayProof
            }
    \item \noindent\mbox{%
            \AxiomC{$\Gamma \vdash M : \Pi x : A . B$}
            \AxiomC{$\Gamma \vdash N : A$}
            \LeftLabel{(\emph{appl})}
            \BinaryInfC{$\Gamma \vdash MN : B[x := N]$}
            \DisplayProof
            }
    \item \noindent\mbox{%
        \AxiomC{$\Gamma, x : A \vdash M : B$}
        \AxiomC{$\Gamma \vdash \Pi x : A . B$}
        \LeftLabel{(\emph{abst})}
        \BinaryInfC{$\Gamma \vdash \lambda x : A . M : \Pi x : A . B$}
        \DisplayProof
    }
    \item \noindent\mbox{%
            \AxiomC{$\Gamma \vdash A : B $}
            \AxiomC{$\Gamma \vdash B' : s$}
            \LeftLabel{(\emph{conv})}
            \RightLabel{se $B =_{\beta} B'$}
            \BinaryInfC{$\Gamma \vdash A : B'$}
            \DisplayProof
    }
\end{itemize}

\subsubsection{Propriedades de \texorpdfstring{$\lambda C$}{LC}}

A maioria das propriedades de $\lambda C$ são extensões das propriedades vistas anteriormente nos seus subsistemas. O que muda de um para o outro é a necessidade de construir definições e proposições que abarquem as diversas dependências que existem no sistema. Muitas das provas também se tornam mais e mais complexas a medida que são adicionadas novas regras e formas de se relacionar os sorts

Primeiro, é necessário definir as expressões de $\lambda C$:

\begin{definition}[Expressões de $\lambda C$, $\mathcal{E}$, \cite{nederpelt2014}]
    O conjunto $\mathcal{E}$ de $\lambda C$-expressões é definido pela seguinte BNF:
    $$\mathcal{E} = V | \square | \ast | (\mathcal{E} \mathcal{E}) | (\lambda V : \mathcal{E} . \mathcal{E}) | \Pi V : \mathcal{E} . \mathcal{E}$$ 
\end{definition}

A notação de parenteses, abstrações sucessivas e aplicações sucessivas é a mesma.

\begin{lemma}[\emph{Lema das Variáveis Livres}, \cite{nederpelt2014}]
    Se $\Gamma \vdash A : B$, então $FV(A)$ e $FV(B) \subseteq dom(\Gamma)$
\end{lemma}

O contexto é dito \emph{bem-formado} se ele forma parte de um juizo derivável:

\begin{definition}[\cite{nederpelt2014}]
    Um contexto $\Gamma$ é \emph{bem formado} se existem $A$ e $B$ tais que $\Gamma \vdash A : B$.
\end{definition}

Com isso, é possível ver os seguintes lemas:

\begin{lemma}[Afinamento, Condensação, Permutação, \cite{nederpelt2014}]
    \hfil
    \begin{enumerate}[label=(\arabic*)]
        \item (\emph{Afinamento}) Sejam $\Gamma '$ e $\Gamma ''$ contextos tais que $\Gamma' \subseteq \Gamma ''$. Se $\Gamma ' \vdash A : B$ e $\Gamma ''$ é bem formado, então $\Gamma '' \vdash A : B$
        \item (\emph{Permutação}) Sejam $\Gamma '$ e $\Gamma ''$ dois contextos e seja $\Gamma ''$ a permutação de $\Gamma '$. Se $\Gamma ' \vdash A : B$ e $\Gamma ''$ é bem formado, então $\Gamma '' \vdash A : B$.
        \item (\emph{Condensação}) Se $\Gamma ', x : A, \Gamma '' \vdash B : C$ e $x$ não ocorre em $\Gamma ''$, $B$ ou $C$, então $\Gamma', \Gamma'' \vdash B : C$.
    \end{enumerate}
\end{lemma}

Também é interessante saber se uma expressão é \emph{legal} ou não:

\begin{definition}[\cite{nederpelt2014}]
    Uma expressão $M$ em $\lambda C$ é \emph{legal} se existe $\Gamma$ e $N$ tais que $\Gamma \vdash M : N$ ou $\Gamma \vdash N : M$ (Ou seja, quando $M$ é \emph{tipável} ou \emph{habitável})
\end{definition}

\begin{lemma}[Lema da subexpressão, \cite{nederpelt2014}]
    Se $M$ é legal, então toda subexpressão de $M$ é legal
\end{lemma}

Outro lema importante:

\begin{lemma}[Lema da unicidade de tipos a menos que conversão, \cite{nederpelt2014}]
    Se $\Gamma \vdash A : B_1$ e $\Gamma \vdash A : B_2$, então $B_1 =_{\beta} B_2$
\end{lemma}

\begin{lemma}[Lema da substituição, \cite{nederpelt2014}]
    Seja $\Gamma ', x : A, \Gamma '' \vdash B : C$ e $\Gamma ' \vdash D : A$, então $\Gamma ', \Gamma '' [x := D] \vdash B[x := D] : C [x := D]$
\end{lemma}

Esse lema fala que é possível substituir $D$ em $\Gamma ', x : A, \Gamma '' \vdash B : C$ caso $D$ e $x : A$ possuam o mesmo tipo. Como tipos também podem possuir termos, essa substituição vai ocorer em todos os lugares.

Um teorema presente também em $\lambda C$ é o Teorema de Church-Rosser:

\begin{theorem}[Teorema de Church-Rosser, \cite{nederpelt2014}]
    A propriedade de Church-Rosser é válida para $\lambda C$, ou seja, se $M$ está em $\mathcal{E}$, $M \twoheadrightarrow_{\beta} N_1$ e $M \twoheadrightarrow_{\beta} N_2$, então existe um $N_3$ tal que $N_1 \twoheadrightarrow_{\beta} N_3$ e $N_2 \twoheadrightarrow_{\beta} N_3$
\end{theorem}

\begin{corollary}[\cite{nederpelt2014}]
    Suponha que $M$ e $N$ estão em $\mathcal{E}$ e $M =_{\beta} N$, então existe um $L$ tal que $M \twoheadrightarrow_{\beta} L$ e $N \twoheadrightarrow_{\beta} L$
\end{corollary}

\begin{lemma}[Redução de sujeito, \cite{nederpelt2014}]
    Se $\Gamma \vdash A : B$ e $A \twoheadrightarrow_{\beta} A'$, então $\Gamma \vdash A' : B$
\end{lemma}

Outro teorema muito importante que está presente em $\lambda C$ é o da normalização forte:

\begin{theorem}[Teorema da Normalização Forte, \cite{nederpelt2014}]
    Todo termo legal $M$ é fortemente normalizável
\end{theorem}

Finalmente, de um ponto de vista computacional, a questão que fica é saber se a Boa-tipagem e a checagem de tipos é decidível ou não em $\lambda C$, e a resposta é sim:

\begin{theorem}[Decidabilidade em $\lambda C$, \cite{nederpelt2014}]
    Em $\lambda C$ e seus subsistemas, as questões de Boa-tipagem e Checagem de tipos são decidíveis
\end{theorem}

Logo, é possível desenvolver um \emph{programa de computador} que resolve esses problemas automaticamente. Um desses programas é o \emph{Provador de Teoremas} \textbf{Coq}.

A questão de Encontrar os termos só é decidível em $\lambda_{\to}$ e $\lambda \underline{\omega}$, mas indecidível nos outros sistemas.

\end{document}
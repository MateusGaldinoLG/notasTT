\documentclass[../main.tex]{subfiles}

\begin{document}


\subsection{História da Teoria dos Tipos}

Para entender um pouco do que é a teoria dos tipos, é necessário entender sua história de desenvolvimento. Para isso, é necessário voltar até sua criação. \newline

\subsubsection{Russel e os problemas da teoria dos conjuntos}

% Expandir essa parte histórica depois

Do final do século XIX até o início do século XX, um dos temas de maior relevância para a matemática era a chamada Teoria dos Conjuntos. A primeira publicação que tratava do conceito de \textbf{conjunto} foi o artigo de \textbf{Georg Cantor} "Über eine Eigenschaft des Inbegriffes aller reellen algebraischen Zahlen" (Sobre uma propriedade da coleção de todos os números reais algébricos), no contexto de estudos em álgebra. Para Cantor, um conjunto (\emph{Menge} ou \emph{Inbegriff}) seria definido como "qualquer combinação $M$ de certos objetos completamente distintos $m$ em nossa intuição [Anschauung] ou em nosso pensamento (que são conhecidos como "Elementos" de $M$) como um todo."


Uma das primeiras pessoas a utilizar o conceito desenvolvido por Cantor para formular uma teoria sobre as bases da matemática seria \textbf{Gottlob Frege} no seu livro "Grundlagen der Arithmetik" (Fundamentos da aritmética). Nesse livro, Frege tenta explicar filosoficamente as bases da matemática e para isso ele se vale da teoria dos conjuntos, também da definição de número feita por \textbf{Giuseppe Peano}.


Um grande leitor das obras de Frege foi o matemático \textbf{Bertrand Russel}. Ele percebeu em seus estudos um problema na teoria dos conjuntos que fazia essa teoria, no seu estado naquela época, ser insuficiente para uma construção sólida das bases da matemática. Esse problema seria um paradoxo desenvolvido através das próprias definições de conjunto encontradas até então. De forma básica, o \emph{Paradoxo de Russell} era construido através da seguinte definição. Seja o conjunto $R$ definito por $R = \{ x | x \notin x \}$, ou seja, o conjunto de todos os conjuntos que não possuem a si mesmos como elemento. A pergunta que fica é $R \in R$? Se sim, pela própria definição de $R$, então $R \notin R$. Mas se não, ou seja $R \notin R$, então, novamente pela definição de $R$, $R \in R$. Logo, chega-se a um paradoxo.

Tendo em vista essa questão, os autores \textbf{Bertrand Russel} e \textbf{Alfred North Whitehead} decidem desenvolver uma obra para a formalização completa da matemática desde suas bases até áreas mais avançadas, chamada de Principia Mathematica. Nessa obra, Russel coloca novamente a questão desse paradoxo, mas dessa vez ele introduz uma possível solução. Nessa solução, ele divide os conjuntos em tipos que só podem possuir elementos de outros tipos. Isso faz com que esses tipos não deem brecha para possuirem a si mesmos.


Dessa forma, a noção de tipos ainda estava vinculada à noção de conjuntos, mas aparecia para resolver problemas relacionados à, como ficou chamada poteriormente, Teoria Ingênua dos conjuntos.


\end{document}
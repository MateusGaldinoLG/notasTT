\documentclass[../main.tex]{subfiles}

\begin{document}

\section{Teoria dos Tipos de Martin-Löf}

Nesse capítulo, o interesse é dar uma revisão geral da teoria dos tipos dependentes e apresentar a Teoria dos tipos de Martin-Löf, uma teoria dos tipos dependentes com tipos adicionais. 

\subsection{Apontamentos iniciais}

Como visto no capítulo 5, a teoria dos tipos dependentes é uma extensão da teoria dos tipos simples que permite a construção de tipos que dependam de termos. Por exemplo, seja $n : \mathbf{N}$ um termo do tipo dos naturais, é possível construir o tipo $\text{Vect } n$ dos vetores de tamanho $n$.

A teoria dos tipos dependentes de Martin-Löf foi chamada por ele também de Teoria dos Tipos Intuicionista, pois é um tipo de teoria dos tipos que se preocupa com a formalização da matemática intuicionista presente em autores como Bishop.

\subsection{Os tipos da MLTT}

A exposição dos tipos presentes na teoria dos tipos de Martin-Löf é bastante correlata à exposição dos tipos presentes na teoria dos tipos simples feita na segunda parte do capítulo 3. Como na teoria dos tipos simples, as regras de cada tipo da Teoria dos Tipos de Martin-Lof são divididas em quatro partes: 

\begin{itemize}
    \item Uma \emph{regra de formação} que diz como formar o tipo
    \item Uma \emph{regra de introdução} que diz como introduzir novos termos do tipo
    \item Uma \emph{regra de eliminação} que diz como usar e remover termos do tipo
    \item Uma \emph{regra de computação} que diz como as regras de introdução e de eliminação se relacionam
\end{itemize}

\subsubsection*{O tipo produto}




\end{document}
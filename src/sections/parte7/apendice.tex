\documentclass[../main.tex]{subfiles}

\begin{document}

\section{Apêndice Histórico}

Esse apêndice histórico serve como uma forma do leitor se localizar nos fatos mais importantes para as áreas do livro. Cada fato vêm com um parentese antes para definir para qual área, ou áreas, o fato é relevante

\begin{itemize}
    \item  1958 - (Teoria dos Tipos) publicação por Kurt Gödel do artigo "Über eine bisher noch nicht benützte Erweiterung des finiten Standpunktes" (Sobre uma extensão do ponto de vista finitário ainda não utilizada) que representa o início do assim chamado "Sistema T".
    \item 1984 - (HoTT, $\infty$-cats) - Publicação por Grothendieck do seu \emph{Esquisse d'un Programme} (Esboço de um programa) onde ele publica a \textbf{Hipótese da Homotopia} de que os $n$-tipos homotópicos podem ser equivalentes ao $n$-grupoide, com $n \in \mathbb{N} \cup \{\infty\}$ 
    \item 1991 - (HoTT, $\infty$-cats) - Kapranov e Voevodsky publicam uma prova de que a categoria homotópica dos espaços é equivalente à categoria homotópica dos $\infty$-grupoides fracos (Hipótese da homotopía para $n = \infty$). Essa prova possuia uma falha e essa falha foi a motivação para que Voevodsky desenvolvesse a Homotopy Type Theory
    \item 1998 - ($\infty$-cats) - Carlos Simpson publica um artigo que possua um contraexemplo ao resultado principal do artigo de 1991 de Kapranov-Voevodsky
\end{itemize}

\end{document}
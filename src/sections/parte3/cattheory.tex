\documentclass[../main.tex]{subfiles}

\begin{document}

\section{Introdução à Teoria das Categorias}

A teoria das categorias é uma área de matemática que relaciona diversas áreas, como por exemplo, Teoria dos Grupos, Teoria dos Anéis, Topologia, Teoria dos Grafos, etc. Cada uma dessas teorias tem em comum a definição de seus objetos (Grupos, Anéis, Espaços topológicos, grafos) e formas de relacionar esses objetos (Homomorfismos de grupos, homomorfismos de aneis, homeomorfismos, homomorfismos entre grafos). 

\subsection{Categorias}

Para estudar categorias, primeiro é necessário defini-las:

\begin{definition}[Categoria, \cite{awodey2010}]
    Uma \emph{categoria} $\mathbf{C}$ consiste em:
    \begin{itemize}
        \item \emph{Objetos}: $A, B, C, \dots$
        \item \emph{Setas} (Morfismos): $f, g, h, \dots$
        \item Para cada seta $f$ existem objetos: $$dom(f), cod(f)$$ chamados de \emph{domínio} e \emph{contradomínio} de $f$. A escrita $$f : A \to B$$ indica que $A = dom(f)$ e $B = cod(f)$
        \item Sejam setas $f : A \to B$ e $g : B \to C$ com: $$cod(f) = dom(g)$$ existe uma seta $g \circ f : A \to C$ chamada de \emph{composição} de $f$ com $g$
        \item Para cada objeto $A$ existe uma seta $$1_A : A \to A$$ chamada de \emph{seta identidade} de $A$
    \end{itemize}
    Esses dados precisam satisfazer os seguintes axiomas:
    \begin{itemize}
        \item (Associatividade) Sejam $f : A \to B$, $g : B \to C$ e $h : C \to D$ setas, então: $$h \circ (g \circ f) = (h \circ g ) \circ f$$
        \item (Identidade) Seja $f : A \to B$ uma seta, então $$f \circ 1_A = f = 1_B \circ f$$
    \end{itemize}
\end{definition}

Para quaisquer objetos $A$ e $B$ em uma categoria $C$, a coleção de setas de $A$ para $B$ é escrito $Hom_C(A, B)$

Alguns exemplos de categorias são:

\begin{enumerate}
    \item A categoria \textbf{Set} que possui conjuntos como objetos e funções como morfismos. 
    \item Os conjuntos ordenados descritos na Definição 1.31 também podem formar uma categoria junto com os mapeamentos monótonos descritos na Definição 1.32, chamada de \textbf{Pos}
    \item Um monóide é um conjunto $M$ equipado com uma operação binária $\cdot : M \times M \to M$ e um elemento unitário $e \in M$ tal que para todo $x, y, z \in M$: $$x \cdot (y \cdot z) = (x \cdot y) \cdot z$$ e $$e \cdot x = x = x \cdot e$$. Por exemplo, o conjunto dos naturais $\mathbb{N}$, junto à operação de soma usual $+ : \mathbb{N} \times \mathbb{N} \to \mathbb{N}$, pode ser considerado um monoide, com o $0$ como elemento unitário. \\
    Dois monóides $(M, \cdot)$ e $(N, \star)$ podem ser relacionados através de um \emph{homomorfismo} $\phi : M \to N$ tal que $$\phi (x \cdot y) = \phi(x) \star \phi(y)$$ e $$\phi(e_M) = e_N$$
    A categoria que possui monóides como objetos e homeomorfismos como morfismos é denominada de \textbf{Mon}
    \item Um grupo $G$ é um monóide onde para todo $a \in G$ existe um elemento $b \in G$ tal que $a \cdot b = e$. $b$ é chamado de \emph{inverso} de $a$ e é escrito como $a^{-1}$. Um homomorfismo $\phi$ entre dois grupos $(G, \cdot)$ e $(H, \star)$ obedece as duas condições para homomorfismos entre monóides mais a seguinte: $$\phi(a^{-1}) = \phi(a)^{-1}$$ A categoria que possui monóides como objetos e homomorfismos como morfismos é denominada de \textbf{Grp}
    \item (\cite{riehl2017}) Um grupo $G$ (e também um monóide) define uma categoria $BG$ com um único objeto. Os elementos do grupo são seus morfismos e a composição é dada por $\cdot$. O elemento unitário $e \in G$ age como o morfismo identidade para o objeto único dessa categoria. \\ Por exemplo, para $(\mathbb{Z}, +)$, $e = 0$ e será representado por $0 : \mathbb{Z} \to \mathbb{Z}$. Sendo $ 1 : \mathbb{Z} \to \mathbb{Z}$ e $2 : \mathbb{Z} \to \mathbb{Z}$, então a composição $1 \circ 2$ é em $(\mathbb{Z}, +)$ equivalente a $1 + 2$ e $1 \circ 2 = 3$.
\end{enumerate}

% \begin{proposition}
%     O sistema $\lambda_{\to}$ é uma categoria que tem:
%     \begin{itemize}
%         \item Objetos: os tipos simples $\alpha, \beta, \gamma, \dots$
%         \item Morfismos: os tipos funcionais entre os tipos simples
%     \end{itemize}
% \end{proposition}

% \emph{Prova}: Primeiro, é claro que os seguintes tipos:
% \begin{itemize}
%     \item $(\alpha \to \beta) \to (\beta \to \gamma) \to \alpha \to \gamma$ (Equivalente à composição)
%     \item $(\alpha \to \alpha)$ (Equivalente à seta identidade)
% \end{itemize} 

% são habitados em $\lambda_{\to}$ para quaisquer tipos $\alpha, \beta, \gamma$. Após isso é necessário provar a associatividade:

% $$(\alpha \to \beta) \to (\beta \to \gamma) \to (\gamma \to \delta) \to \alpha \to \gamma$$

% O que equivale a provar que esse tipo é habitado. Mas isso pode ser feito usando o seguinte termo:

% $$\lambda f : (\alpha \to \beta) . \lambda g : (\beta \to \gamma) . \lambda h : (\gamma \to \delta) . \lambda a : \alpha . hgfa$$

% Para a identidade, é o mesmo que

\begin{definition}[Isomorfismos, \cite{awodey2010}]
    Em qualquer categoria $C$, um morfismo $f : A \to B$ é chamado de \emph{isomorfismo} se existe um morfismo $g : B \to A$ em $C$ tal que $$g \circ f = 1_A \text{ e } f \circ g = 1_B$$
\end{definition}

$g$ é chamado de inverso de $f$ e, por ser único, pode ser denotado por $f^{-1}$. Os objetos $A$ e $B$ são ditos \emph{isomórficos} e denotados por $A \cong B$

Exemplos:

\begin{enumerate}
    \item Os isomorfismos em \textbf{Set} são bijeções
    \item Os isomorfismos em \textbf{Grp} são os homomorfismos bijetivos
\end{enumerate}

\begin{definition}[Categorias pequenas, \cite{awodey2010}]
    Uma categoria $C$ é chamada de \emph{pequena} se a coleção $C_0$ de objetos em $C$ e a coleção $C_1$ de morfismos em $C$ são conjuntos. Caso contrário, $C$ é chamada de \emph{grande}
\end{definition}

Todas as categorias finitas são pequenas, assim como a categoria \textbf{$Sets_{fin}$} de conjuntos finitos. Já a categoria \textbf{Sets} é grande (Pois caso a coleção de seus objetos fosse um conjunto, isso geraria o paradoxo de Russell)

\begin{definition}[Categoria localmente pequena, \cite{awodey2010}]
    Uma categoria $C$ é chamada de \emph{localmente pequena} se para quaisquer objetos $X$ e $Y$ em $C$, a coleção de morfismos $Hom_C(X, Y) = \{f \in C_1 | f : X \to Y\}$ é um \emph{conjunto} (Chamado de \emph{hom-set})
\end{definition}

\subsection{Categorias novas das antigas}

Dada a definição de categorias, é interessante analisar o que pode ser feito com uma categoria e como gerar novas categorias de categorias antigas

\begin{definition}[Categoria oposta, \cite{awodey2010}]
    A categoria \emph{oposta} (ou "dual") $C^{op}$ de uma categoria $C$ possui os mesmos objetos que $C$, mas para cada morfismos $f : A \to B$ em $C$ existe um morfismo $f : B \to A$ em $C^{op}$
\end{definition}

A categoria oposta inverte todos os morfismos da categoria que parte. Então seja $f^{op}$ o morfismo invertido, a composição na categoria oposta se torna: $f^{op} \circ g^{op} = (g \circ f)^{op}$

É interessante perceber que cada resultado na Teoria das Categorias terá um resultado dual ganho "de graça" ao fazer esse resultado nas categorias duais.

Também é possível ver que $(C^{op})^{op} = C$

\begin{definition}[Categoria de setas, \cite{rosiak2022}]
    Seja uma categoria $C$, definimos a \emph{categoria de setas} de $C$, denotada por $C^{\to}$, tendo:
    \begin{itemize}
        \item Objetos: morfismos $A \xrightarrow{f} B$ de $C$
        \item Morfismos: a partir de um objeto de $C^{\to}$ $A \xrightarrow{f} B$ para outro $A' \xrightarrow{f'} B'$ um morfismo é um par $\langle A \xrightarrow{f} B, A' \xrightarrow{f'} B' \rangle$ de morfismos de $C$ fazendo o diagrama
        \[\begin{tikzcd}
            A && {A'} \\
            \\
            B && {B'}
            \arrow["h", from=1-1, to=1-3]
            \arrow["f"', from=1-1, to=3-1]
            \arrow["{f'}", from=1-3, to=3-3]
            \arrow["k"', from=3-1, to=3-3]
        \end{tikzcd}\]
        comutar. Ou seja, $k \circ f = f' \circ h$ em $C$
    \end{itemize}

    A composição das setas é feita ao colocar quadrados comutativos lado a lado da seguinte forma:
    \[\begin{tikzcd}
        A && {A'} && {A''} \\
        \\
        B && {B'} && {B''}
        \arrow["h", from=1-1, to=1-3]
        \arrow["f"', from=1-1, to=3-1]
        \arrow["l", from=1-3, to=1-5]
        \arrow["{f'}", from=1-3, to=3-3]
        \arrow["{f''}", from=1-5, to=3-5]
        \arrow["k"', from=3-1, to=3-3]
        \arrow["m", from=3-3, to=3-5]
    \end{tikzcd}\]

    tal que $\langle l, m \rangle \circ \langle h, k \rangle = \langle l \circ h, m \circ k \rangle$
    \\
    A identidade de um objeto $A \xrightarrow{f} B$ é dado pelo par $\langle id_A, id_B \rangle$

\end{definition}

Outro tipo de categoria de interesse é a categoria slice:

\begin{definition}[Categoria Slice, \cite{awodey2010}]
    A categoria slice $\textbf{C}/C$ de uma categoria $\textbf{C}$ sobre um objeto $C \in \textbf{C}$ possui:
    \begin{itemize}
        \item Objetos: todas as setas $f \in \textbf{C}$ tal que $cod(f) = C$
        \item Morfismos: $g$ de $f : X \to C$ e $f' : X' \to C$ é uma seta $g : X \to X'$ em $\textbf{C}$ tal que $f' \circ g = f$ como no diagrama:
        \[\begin{tikzcd}
            X && {X'} \\
            & C
            \arrow["g", from=1-1, to=1-3]
            \arrow["f"', from=1-1, to=2-2]
            \arrow["{f'}", from=1-3, to=2-2]
        \end{tikzcd}\]
    \end{itemize}
    A composição desses morfismos é basicamente a junção de desses triangulos \\

    Também é possível definir a categoria $(C / \textbf{C})$ chamada de categoria de co-slice, onde os objetos são setas $f$ de $\textbf{C}$ tal que $dom(f) = C$ e uma seta entre $f : C \to X$ e $f' : C \to X'$ é uma seta $h : X \to X'$ tal que $h \circ f = f'$ como no diagrama:
    \[\begin{tikzcd}
        & C \\
        X && {X'}
        \arrow["f", from=2-1, to=1-2]
        \arrow["g", from=2-1, to=2-3]
        \arrow["{f'}"', from=2-3, to=1-2]
    \end{tikzcd}\]
\end{definition}

Também é possível definir a noção de subcategoria:

\begin{definition}[Subcategoria, \cite{rosiak2022}]
    Uma categoria $\textbf{D}$ dita \emph{subcategoria} de $\textbf{C}$ é obtida restringindo a coleção de objetos de $\textbf{C}$ para uma subcoleção (Ou seja, todo $\textbf{D}$-objeto é um $\textbf{C}$-objeto) e a coleção de morfismos é obtida restringindo a coleção de morfismos de $\textbf{C}$ onde:
    \begin{itemize}
        \item Se o morfismo $f : A \to B$ está em $\textbf{D}$, então $A$ e $B$ estão em $\textbf{D}$
        \item Se $A$ está em $\textbf{D}$, então também está o morfismo identidade $id_A$
        \item Se $f : A \to B$ e $g : B \to C$ estão em $\textbf{D}$, então $g \circ f : A \to C$ também está
    \end{itemize}
\end{definition}

e também:

\begin{definition}[Subcategoria cheia, \cite{rosiak2022}]
    Seja $\textbf{D}$ uma subcategoria de $\textbf{C}$. ENtão $\textbf{D}$ é uma \emph{subcategoria cheia} de \textbf{C} quando \textbf{C} não possui setas $A \to B$ além dos que já existem em $\textbf{D}$. Ou seja para quaisquer objetos $A$ e $B$ em $\textbf{D}$, $\textbf{C}$: $$Hom_{\textbf{D}}(A, B) = Hom_{\textbf{C}}(A, B)$$
\end{definition}

Exemplo:

\begin{itemize}
    \item A categoria \textbf{FinSet} de conjuntos finitos é uma subcategoria de \textbf{Set}.
    \item Um grupo $(G, \cdot)$ é dito \emph{abeliano}, ou comutativo, caso para quaisquer dois elementos $a, b \in G$, $a \cdot b = b \cdot a$. A categoria de grupos abelianos \textbf{Ab} é uma subcategoria (cheia) de \textbf{Grp}
\end{itemize}

Um produto de dois conjuntos $A$ e $B$ é dado por $$A \times B = \{ (a, b) | a \in A, b \in B \}$$. O conjunto produto possui duas projeções:

\[\begin{tikzcd}
	A && {A \times B} && B
	\arrow["{\pi_1}", from=1-3, to=1-1]
	\arrow["{\pi_2}"', from=1-3, to=1-5]
\end{tikzcd}\]

tais que $$\pi_1(a,b) = a, \quad \pi_2(a, b) = b$$.

Dado um elemento $c \in A \times B$, tem-se que $$c = (\pi_1 c, \pi_2 c)$$ capturado no seguinte diagrama:

\[\begin{tikzcd}
	&& 1 \\
	\\
	A && {A \times B} && B
	\arrow["a"', from=1-3, to=3-1]
	\arrow["{(a, b)}"{description}, dashed, from=1-3, to=3-3]
	\arrow["b", from=1-3, to=3-5]
	\arrow["{\pi_1}", from=3-3, to=3-1]
	\arrow["{\pi_2}"', from=3-3, to=3-5]
\end{tikzcd}\]

Para categorias gerais:

\begin{definition}[\cite{awodey2010}]
    Em uma categoria $\mathcal{C}$, um \emph{diagrama de produto} para objetos $A$ e $B$ consiste de um objeto $P$ e setas:

    \[\begin{tikzcd}
        A && P && B
        \arrow["{p_1}", from=1-3, to=1-1]
        \arrow["{p_2}"', from=1-3, to=1-5]
    \end{tikzcd}\]

    Tais que para qualquer diagrama:

    \[\begin{tikzcd}
        A && X && B
        \arrow["{x_1}", from=1-3, to=1-1]
        \arrow["{x_2}"', from=1-3, to=1-5]
    \end{tikzcd}\]

    existe uma seta única $u : X \to P$ fazendo o diagrama

    \[\begin{tikzcd}
        && X \\
        \\
        A && P && B
        \arrow["{x_1}"', from=1-3, to=3-1]
        \arrow["u", dashed, from=1-3, to=3-3]
        \arrow["{x_2}", from=1-3, to=3-5]
        \arrow["{p_1}", from=3-3, to=3-1]
        \arrow["{p_2}"', from=3-3, to=3-5]
    \end{tikzcd}\]

    comutar, ou seja $x_1 = p_1 u$ e $x_2 = p_2 u$
\end{definition}

Seja $\mathcal{C}$ uma categoria com diagramas de produto para cada par de objetos, tem-se então:

\[\begin{tikzcd}
	A && {A \times A'} && {A'} \\
	\\
	B && {B \times B'} && {B'}
	\arrow["f"', from=1-1, to=3-1]
	\arrow["{p_1}"', from=1-3, to=1-1]
	\arrow["{p_2}", from=1-3, to=1-5]
	\arrow["{f'}", from=1-5, to=3-5]
	\arrow["{q_1}", from=3-3, to=3-1]
	\arrow["{q_2}"', from=3-3, to=3-5]
\end{tikzcd}\]

Então, pode-se encontrar um morfismo $$f \times f' : A \times A' \to B \times B'$$ para $f \times f' = \langle f \circ p_1, f' \circ p_2 \rangle$ que faz os dois lados do seguinte diagrama comutarem:

\[\begin{tikzcd}
	A && {A \times A'} && {A'} \\
	\\
	B && {B \times B'} && {B'}
	\arrow["f"', from=1-1, to=3-1]
	\arrow["{p_1}"', from=1-3, to=1-1]
	\arrow["{p_2}", from=1-3, to=1-5]
	\arrow["{f \times f'}", dashed, from=1-3, to=3-3]
	\arrow["{f'}", from=1-5, to=3-5]
	\arrow["{q_1}", from=3-3, to=3-1]
	\arrow["{q_2}"', from=3-3, to=3-5]
\end{tikzcd}\]

Uma categoria que possui um produto para cada par de objetos é dita \emph{ter produtos binários}. Essa construção também pode ser feita para produtos ternários e generalizada para produtos $I$-ários


\subsection{Funtores}

Sendo categorias estruturas que se iniciam com objetos e morfismos entre esses objetos, é natural se perguntar se existem morfismos entre categorias. Esses morfismos deveriam também manter a estrutura entre categorias, como por exemplo os morfismos entre grupos mantém a estrutura do grupo, mesmo que mudando-se os objetos e os morfismos internos à categoria. Esse tipo de morfismo entre categorias é chamado de \emph{funtor} e definido da seguinte forma:

\begin{definition}[Funtor, \cite{awodey2010}]
    Um funtor $F : \mathcal{C} \to \mathcal{D}$ entre categorias $\mathcal{C}$ e $\mathcal{D}$ é um mapeamento de objetos em objetos e setas em setas tal que:
    \begin{enumerate}
        \item $F(f : A \to B) = F(f) : F(A) \to F(B)$
        \item $F(g \circ f) = F(g) \circ F(f)$
        \item $F(1_A) = 1_{F(A)}$
    \end{enumerate}
\end{definition}

A primeira parte define que para cada objeto $A$ em $\mathcal{C}$, existe um objeto correspondente $F(A)$ em $\mathcal{D}$. Também define que Funtores preservam os domínios e codomínios de cada morfismo.

\[\begin{tikzcd}
	A && B &&& {F(A)} && {F(B)} \\
	\\
	&& C &&&&& {F(C)} \\
	& {\mathcal{C}} &&&&& {\mathcal{D}}
	\arrow["f", from=1-1, to=1-3]
	\arrow["{g \circ f}"', from=1-1, to=3-3]
	\arrow["g", from=1-3, to=3-3]
	\arrow["{F(f)}"', from=1-6, to=1-8]
	\arrow["{F(g) \circ F(f)}"', from=1-6, to=3-8]
	\arrow["{F(g)}"', from=1-8, to=3-8]
	\arrow["F"', from=4-2, to=4-7]
\end{tikzcd}\]
% \[\begin{tikzcd}
% 	A && {F(A)} \\
% 	\\
% 	B && {F(B)} \\
% 	\\
% 	C && {F(C)} \\
% 	C && {\mathcal{D}}
% 	\arrow["{1_A}"{pos=0.6}, from=1-1, to=1-1, loop, in=150, out=210, distance=5mm]
% 	\arrow["f", from=1-1, to=3-1]
% 	\arrow["{g \circ f}"'{pos=0.6}, curve={height=-24pt}, from=1-1, to=5-1]
% 	\arrow["{1_{F(a)}}", from=1-3, to=1-3, loop, in=150, out=210, distance=5mm]
% 	\arrow["{F(f)}", from=1-3, to=3-3]
% 	\arrow["{F(g) \circ F(f)}", curve={height=-24pt}, from=1-3, to=5-3]
% 	\arrow["g", from=3-1, to=5-1]
% 	\arrow["{F(g)}", from=3-3, to=5-3]
% 	\arrow["F", from=6-1, to=6-3]
% \end{tikzcd}\]

A segunda parte define que existindo uma composição de morfismos em $\mathcal{C}$, também existira uma composição correspondente em $mathcal{D}$.

A terceira parte define que as identidades também são preservadas.

Em algumas partes da matemática porém, os funtores não preservam a ordem dos morfismos. Os funtores que preservam como da regra 1 da parte anterior são chamados de \emph{funtores covariantes}. É interessante também definir os funtores \emph{contravariantes}:

\begin{definition}[Funtor Contravariante, \cite{awodey2010}]
    Um funtor da forma $F : \mathcal{C}^{op} \to \mathcal{D}$ é chamado de \emph{funtor contravarinate} em $\mathcal{C}$. Ou seja, as regras se tornam:
    \begin{enumerate}
        \item Seja $f : A \to B$ em $\mathcal{C}$, então: $F(f : A \to B) : F(B) \to F(A)$
        \item Seja $g \circ f$ em $\mathcal{C}$, então $F(g \circ f) = F(f) \circ F(g)$
        \item $F(1_A) = 1_{F(A)}$
    \end{enumerate}  
\end{definition}

Um exemplo essencial de funtor contravariante são os \emph{pré-feixes}, definidos da seguinte forma:

\begin{definition}[Pré-feixe, \cite{rosiak2022}]
    Um \emph{pré-feixe} (com valor de conjunto) em $\mathcal{C}$, onde $\mathcal{C}$ é uma categoria pequena, é um funtor $\mathcal{C}^{op} \to \textbf{Set}$
\end{definition}


O pré-feixe pode ser visto como uma atribuição de dados locais de acordo com a estrutura de $\mathcal{C}$.

Uma vez definidos funtores, o próximo passo é se perguntar se existe uma categoria que usa funtores como morfismos entre seus objetos, e a resposta é que existe tal categoria, onde objetos são categorias, definida da seguinte forma:

\begin{definition}[\textbf{Cat}, \cite{rosiak2022}]
    A categoria de \emph{categorias pequenas}, denotada por \textbf{Cat}, é a categoria que possui:
    \begin{itemize}
        \item objetos: categorias pequenas
        \item morfismos: funtores entre elas
    \end{itemize}
\end{definition}

Para demonstrar que essa é de fato uma categoria, é necessário definir um morfismo/funtor identidade e a ideia de composição entre morfismos/funtores.

\begin{definition}[\cite{rosiak2022}]
    Dada uma categoria $\mathcal{C}$, o \emph{funtor identidade} é o funtor $id_{\mathcal{C}} : \mathcal{C} \to \mathcal{C}$ que faz o esperado: leva um objeto a ele mesmo e um morfismos a ele mesmo tal que:
    \begin{itemize}
        \item $id_{\mathcal{C}} (c) = c$
        \item $id_{\mathcal{C}} (f) = f$
    \end{itemize}
\end{definition}

Para a composição, sejam $\mathcal{C}$, $\mathcal{D}$ e $\mathcal{E}$ categorias pequenas, e $F : \mathcal{C} \to \mathcal{D}$ e $G : \mathcal{D} \to \mathcal{E}$ dois funtores, a composição de $G$ com $F$, o funtor composição $G \circ F : \mathcal{C} \to \mathcal{E}$, é definido tal que para objetos $c$ em $\mathcal{C}$, $(G \circ F) (c) = G(F(c))$ e para um morfismo $f : c \to c'$ em $\mathcal{C}$ $(G \circ F) (f) = G(F(f))$. Para definir que $G \circ F$ é um funtor é necessário pedir sua \emph{funtorialidade}, ou seja, que ele obedeça às regras impostas na definição de funtores:

(1) Sejam $f, g$ funtores em $\mathcal{C}$ tal que $g \circ f$ também está definido em $\mathcal{C}$, então: 

\begin{equation*}
    \begin{split}
        (G \circ F) (g \circ f) & = G(F(g \circ f))
                                     \\ & = G(F(g) \circ F(f))
                                     \\ & = G(F(g)) \circ G(F(f))
                                     \\ & = (G \circ F) (g) \circ (G \circ F) (f)
    \end{split}
\end{equation*}

onde a primeira e a última linha são a definição da composição de funtores e o méio é a derivação dessa composição.

(2) Seja $c$ qualquer objeto em $\mathcal{C}$, então:

\begin{equation*}
    \begin{split}
        (G \circ F) (1_c) & = G(F(1_C))
                        \\ & = G(1_{F(c)})
                        \\ & = 1_{G(F(c))}
                        \\ & = 1_{(G \circ F)(c)}
    \end{split}
\end{equation*}

A seguir estão alguns exemplos de funtores:

\begin{enumerate}
    \item Ao tratar monoides como categorias por si só, é interessante entender o que seria equivalente a funtores nesse caso. Normalmente, as relações entre monoides são \emph{homomorfismos de monoides}. Sejam dois monoides $(M, e, \cdot)$ e $(N, e', \star)$, um homomorfismo $\phi : M \to N$ é um mapeamento tal que $$\phi(m \cdot m') = \phi(m) \star \phi(m')$$ e $$\phi(e) = e'$$
    É possível ver que $\phi$ possui a estrutura de funtor entre monoides. $\phi$ seria contravariante se $\phi(m \cdot m') = \phi(m') \star \phi(m)$
    \item Existe um funtor $Core : \textbf{Mon} \to \textbf{Grp}$ que pega um monoide e retorna um subconjunto desse monoide que possui elementos inversos, o que faz com que o monoide se torne um grupo, chamado de \emph{cerne} (\emph{Core}) do monoide.
    \item Existe um funtor $U : \textbf{Grp} \to \textbf{Mon}$ chamado de \emph{Forgetful functor} (Funtor esquecido) que pega um grupo e retorna o monoide correspondente, "esquecendo" a estrutura a mais que caracteriza os grupos.
    \item Outro funtor esquecido é $U : \textbf{Cat} \to \textbf{Grph}$ que pega cada categoria e retorna o grafo correspondente a ela. Um Grafo $G = (V, A, s, t)$ é composto de um conjunto de vertices $V$, um conjunto de arestas $A$ que são direcionadas, e um par de funções $s, t : A \to B$ que codifica a direção das arestas ao assinalar a cada aresta $a \in A$ um inicio $s(a) \in V$ e um fim $t(a) \in V$.
\end{enumerate}

A coleção de objetos de uma categoria $\mathcal{C}$ será denotada por $\mathcal{C}_0$ e a coleção de morfismos será denotada por $\mathcal{C}_1$ na próxima definição:

\begin{definition}[\cite{awodey2010}]
    Um funtor $F : \mathcal{C} \to \mathcal{D}$ é dito:
    \begin{itemize}
        \item \emph{injetivo em objetos} se a parte de objetos $F_0 : \mathcal{C}_0 \to \mathcal{D}_0$ é injetiva, é \emph{sobrejetora em objetos} se $F_0$ é sobrejetora
        \item De forma similar, $F$ é \emph{injetiva} (resp. \emph{sobrejetora}) em \emph{setas} se a parte de setas $F_1 : \mathcal{C}_1 \to \mathcal{D}_1$ é injetiva (resp. sobrejetora)
        \item $F$ é dito \emph{fiel} (Faithful) se, para todo $A, B \in \mathcal{C}_0$, o mapa $F_{A, B} : Hom_{\mathcal{C}}(A, B) \to Hom_{\mathcal{D}}(FA, FB)$ definido por $f : F(f)$ é injetivo
        \item Similarmente $F$ é dito \emph{cheio} (full) se $F_{A, B}$ é sempre sobrejetor
    \end{itemize}
\end{definition}

Exemplo: Seja o funtor esquecido $U : \textbf{Grp} \to \textbf{Set}$. $U_0$ é basicamente o mapeamento dos conjuntos que forma o grupo para os próprios conjuntos, logo $U_0$ é injetivo e sobrejetor em objetos. $U_1$ mapeia os homomorfismos de grupo para as funções correspondentes. Mas dois homomorfismos de grupo com o mesmo domínio e codomínio são iguais se são dados pelas mesmas funções nos conjuntos internos. Porém, nem todas as funções em $\textbf{Set}$ são mapeadas por homomorfismos, logo $U_1$ é injetivo em setas mas não sobrejetor em setas. Os motivos para dizer que $U_1$ é injetor em setas podem ser usados para mostrar que $U$ é fiel. 

\subsection{Transformações Naturais}

Uma vez tendo definido morfismos entre categorias, se torna possível pensar morfismos entre esses morfismos. No caso, morfismos entre funtores:

\begin{definition}[Transformação Natural, \cite{rosiak2022}]
    Sejam duas categorias $\mathcal{C}$ e $\mathcal{D}$ e funtores $F, G : \mathcal{C} \to \mathcal{D}$. Uma \emph{Transformação Natural} $\alpha : F \Rightarrow G$ representado em relação a seus dados como: 
    % \[\begin{tikzcd}
    %     C && D
    %     \arrow[""{name=0, anchor=center, inner sep=0}, "F", curve={height=-12pt}, from=1-1, to=1-3]
    %     \arrow[""{name=1, anchor=center, inner sep=0}, "G"', curve={height=12pt}, from=1-1, to=1-3]
    %     \arrow["\alpha", shorten <=3pt, shorten >=3pt, Rightarrow, from=0, to=1]
    % \end{tikzcd}\]
    Consiste no seguinte:
    \begin{itemize}
        \item Para cada objeto $c \in \mathcal{C}$, um morfismo $\alpha_c : F(c) \to G(c)$ em $\mathcal{D}$ chamado de $c$-componente de $\alpha$, a coleção do qual (para todo objeto em $\mathcal{C}$) define os \emph{componentes} da transformação natural
        \item Para cada morfismo $f : c \to c'$ em $\mathcal{C}$ o seguinte quadrado de morfismos, chamado de \emph{quadrado de naturalidade} de $f$, que deve comutar em $\mathcal{D}$:
        \[\begin{tikzcd}
            c && {F(c)} && {G(c)} \\
            \\
            {c'} && {F(c')} && {G(c')}
            \arrow["f"', from=1-1, to=3-1]
            \arrow["{\alpha_{c}}"', from=1-3, to=1-5]
            \arrow["{F(f)}"', from=1-3, to=3-3]
            \arrow["{G(f)}", from=1-5, to=3-5]
            \arrow["{\alpha_{c'}}"', from=3-3, to=3-5]
        \end{tikzcd}\]
    \end{itemize}
    A coleção de transformações naturais entre $F$ e $G$ é por vezes denotada por $Nat(F, G)$
\end{definition}

É dito que morfismos em uma categoria possuem \emph{naturalidade} quando possuem um comportamento parecido com o do \emph{quadrado de naturalidade}, ou seja se $G(f) \circ \alpha_c = \alpha_{c'} \circ F(f)$.

Uma vez que as transformações naturais ajudam a comparar dois funtores entre si, é interessante saber quando os dois funtores são praticamente iguais. Para isso, vamos usar a seguinte definição:

\begin{definition}[Isomorfismo natural, \cite{rosiak2022}]
    Um \emph{isomorfismo natural} é uma transformação natural $\alpha : F \Rightarrow G$ para qual todo componente $\alpha_c : F(c) \to G(c)$ em $\mathcal{D}$ é um isomorfismo (na categoria alvo). Ou seja, cada $\alpha_c$ possui um inverso $\alpha_c^{-1} : G(c) \to F(c)$ onde os inversos formam componentes de uma transformação natural $\alpha^{-1}$ de $G$ para $F$.
\end{definition}

Se $\alpha$ for um isomorfismo, usa-se a notação $\alpha : F \cong G$

Uma vez definida a equivalência entre funtores, é interessante definir equivalência entre categorias:

\begin{definition}[equivalência de categorias, \cite{rosiak2022}]
    Uma \emph{equivalência de categorias} consiste de um par de funtores $F : \mathcal{C} \to \mathcal{D}$ e $G : \mathcal{D} \to \mathcal{C}$ junto com os isomorfismos naturais $\eta : id_{\mathcal{C}} \cong G \circ F$ e $\epsilon : F \circ G \cong id_{\mathcal{D}}$. Outro jeito de dizer isso é que os funtores são inversos entre si "até o isomorfismo natural de funtores". As categorias $\mathcal{C}$ e $\mathcal{D}$ são ditas \emph{equivalentes} se existe uma equivalência de categorias entre elas, isso é denotado por $\mathcal{C} \simeq \mathcal{D}$
\end{definition}

Uma construção interessante é a categoria de setas que possui funtores como objetos e transformações naturais como morfismos, definida como:

\begin{definition}[\cite{rosiak2022}]
    Para qualquer par fixo de categorias $\mathcal{C}$ e $\mathcal{D}$, pode-se formar uma \emph{categoria de funtores} denotada por $\mathcal{D}^{\mathcal{C}}$ (Ou também $Fun(\mathcal{C}, \mathcal{D})$) que possui:
    \begin{itemize}
        \item objetos: todos os funtores de $\mathcal{C}$ para $\mathcal{D}$
        \item morfismos: todas as transformações naturais entre tais funtores
    \end{itemize}
\end{definition}

Para demonstrar o aspecto de morfismo das transformações naturais, é necessário definir a transformação natural identidade, dada simplismente por $id_F : F \Rightarrow F$, e a composição entre transformações naturais, dada pela seguinte definição:

\begin{definition}[\cite{rosiak2022}]
    Sejam $\alpha : F \Rightarrow G$ e $\beta : G \Rightarrow H$ transformações naturais entre os funtores paralelos $F, G, H$ entre $\mathcal{C}$ e $\mathcal{D}$ como no seguinte diagrama:
    % \[\begin{tikzcd}
    %     C &&& D
    %     \arrow[""{name=0, anchor=center, inner sep=0}, "H"', curve={height=30pt}, from=1-1, to=1-4]
    %     \arrow[""{name=1, anchor=center, inner sep=0}, "F", curve={height=-30pt}, from=1-1, to=1-4]
    %     \arrow[""{name=2, anchor=center, inner sep=0}, "G"{description}, from=1-1, to=1-4]
    %     \arrow["\alpha", shorten <=4pt, shorten >=4pt, Rightarrow, from=1, to=2]
    %     \arrow["\beta", shorten <=4pt, shorten >=4pt, Rightarrow, from=2, to=0]
    % \end{tikzcd}\]
    Existe uma transformação natural $\beta \circ \alpha : F \Rightarrow H$, definida em cada componente como: $(\beta \circ \alpha)_c := \beta_c \circ \alpha_c$ dada pela composição de $\beta$ e $\alpha$.
\end{definition}

Esse estilo de composição é denominado de \emph{compisição vertical}

Já a composição horizontal denotada pelo simbolo $\diamond$ dado por $\beta \diamond \alpha : F_2 \circ F_1 \Rightarrow G_2 \circ G_1$, os quais cada componente em $c$ de $\mathcal{C}$ é definido como o composto do seguinte diagrama comutativo:

% \[\begin{tikzcd}
% 	C &&& D && E
% 	\arrow[""{name=0, anchor=center, inner sep=0}, "{G_1}"', curve={height=30pt}, from=1-1, to=1-4]
% 	\arrow[""{name=1, anchor=center, inner sep=0}, "{F_1}", curve={height=-30pt}, from=1-1, to=1-4]
% 	\arrow[""{name=2, anchor=center, inner sep=0}, "{F_2}", curve={height=-30pt}, from=1-4, to=1-6]
% 	\arrow[""{name=3, anchor=center, inner sep=0}, "{G_2}", curve={height=30pt}, from=1-4, to=1-6]
% 	\arrow["\alpha", shorten <=8pt, shorten >=8pt, Rightarrow, from=1, to=0]
% 	\arrow["\beta", shorten <=8pt, shorten >=8pt, Rightarrow, from=2, to=3]
% \end{tikzcd}\]

\subsection{Limites}

Antes de definir limites, é necessário definir o que são diagramas em uma categoria:

\begin{definition}[Diagrama, \cite{riehl2017}]
    Um \emph{Diagrama} em uma categoria $\mathcal{C}$ é um funtor $F : \mathcal{J} \to \mathcal{C}$ que possui como domínio, chamado de \emph{Categoria Índice} ou \emph{template}, uma categoria pequena
\end{definition}

Um diagrama é normalmente pensado como um grafo orientado. Um diagrama pode ser pensado como a instanciação ou realização em $\mathcal{C}$ de um template específico $\mathcal{J}$.

Seja 

\[\begin{tikzcd}
	& \bullet \\
	\bullet && \bullet
	\arrow[from=1-2, to=2-3]
	\arrow[from=2-1, to=1-2]
	\arrow[from=2-1, to=2-3]
\end{tikzcd}\]

O template do seguinte diagrama:

\[\begin{tikzcd}
	& B \\
	A && C
	\arrow[from=1-2, to=2-3]
	\arrow[from=2-1, to=1-2]
	\arrow[from=2-1, to=2-3]
\end{tikzcd}\]

na categoria $\mathcal{C}$. Seja qualquer objeto $X$ em $\mathcal{C}$, existe um \emph{Funtor constante} denotado também por $X$ que leva todo objeto para $X$ e todo morfismo para a seta identidade em $X$. Então $X$ é, em si, um diagrama $X : \mathcal{J} \to \mathcal{C}$.

Como $X$ e $F$ são dois funtores, então vai pode ser construida uma transformação natural entre eles. Essa tranformação natural vai consistir na coleção de morfismos em $X$ e objetos encontrados no diagrama $F$ tal que esses morfismos comutam com os morfismos encontrados no diagrama. Quando esses morfismos vão de $X$ para $F$, a construção é chamada de \emph{Cone} e, no exemplo é da seguinte forma:

% \[\begin{tikzcd}
% 	& X \\
% 	& B \\
% 	A && C
% 	\arrow[from=1-2, to=2-2]
% 	\arrow[curve={height=12pt}, from=1-2, to=3-1]
% 	\arrow[curve={height=-12pt}, from=1-2, to=3-3]
% 	\arrow[from=2-2, to=3-3]
% 	\arrow[from=3-1, to=2-2]
% 	\arrow[from=3-1, to=3-3]
% \end{tikzcd}\]

Já indo os morfismos de $F$ para $X$, essa construção é chamada de \emph{Cocone} e, seguindo o exemplo anterior, é da seguinte forma:

% \[\begin{tikzcd}
% 	& B \\
% 	A && C \\
% 	& X
% 	\arrow[from=1-2, to=2-3]
% 	\arrow[from=1-2, to=3-2]
% 	\arrow[from=2-1, to=1-2]
% 	\arrow[from=2-1, to=2-3]
% 	\arrow[curve={height=12pt}, from=2-1, to=3-2]
% 	\arrow[curve={height=-12pt}, from=2-3, to=3-2]
% \end{tikzcd}\]


Um limite em um diagrama $F$ é um caso especial de cone sobre $F$, onde o cone se aproxima o máximo possível do diagrama $F$.

Para definir um limite, é necessário fazer algumas definições primeiro:

\begin{definition}[Objetos terminais e iniciais, \cite{awodey2010}]
    Em uma categoria $\mathcal{C}$, um objeto
    \begin{itemize}
        \item $0$ é dito \emph{inicial} se para qualquer objeto $C$ de $\mathcal{C}$ existe um morfismo único $$0 \to C$$
        \item $1$ é dito \emph{terminal} se para qualquer objeto $C$ existe um morfismo único $$C \to 1$$
    \end{itemize}
\end{definition}

\begin{proposition}[\cite{awodey2010}]
    Objetos iniciais (terminais) são únicos a menos que isomorfismo
\end{proposition}

\emph{Prova}: Se $C$ e $C'$ são ambos iniciais (finais), então existe um isomorfismo \emph{único} $C \to C'$.

Sejam $0$ e $0'$ ambos objetos iniciais e seja o seguinte diagrama:

\[\begin{tikzcd}
	0 && {0'} \\
	\\
	&& 0 && {0'}
	\arrow["u", from=1-1, to=1-3]
	\arrow["{1_{0}}", from=1-1, to=3-3]
	\arrow["v", from=1-3, to=3-3]
	\arrow["{1_{0'}}", from=1-3, to=3-5]
	\arrow["u", from=3-3, to=3-5]
\end{tikzcd}\]

é possível ver que $u \circ v = 1_0$ e $v \ circ u = 1_{0'}$, logo eles são unicamente isomórficos. $\Box $
 
Exemplos:

\begin{itemize}
    \item Em \textbf{Sets}, o conjunto vazio é o conjunto inicial e o conjunto unitário é o conjunto terminal. Ou seja, \textbf{Sets} possui um único conjunto inicial mas vários conjuntos terminais.
    \item Em \textbf{Cat}, a categoria $\textbf{0}$ (com nenhum objeto nem nenhum morfismo) é inicial, enquanto a categoria $\textbf{1}$ é terminal.
    \item Em \textbf{Grp} o grupo com único elemento é tanto inicial quanto terminal, assim como em \textbf{Mon}
\end{itemize}

Seja $t : \mathcal{J} \to \textbf{1}$ o funtor único para a categoria terminal. Seja uma categoria $\mathcal{C}$ com um objeto $X \in Ob(\mathcal{C})$. Esse objeto pode ser representado pelo funtor $X : \textbf{1} \to \mathcal{C}$. Então fazendo a composição de $X$ em $t$, tem-se $X \circ t : \mathcal{J} \to \mathcal{C}$ que da o \emph{funtor constante} em $X$, que envia cada objeto em $\mathcal{J}$ para o mesmo $\mathcal{C}$-objeto $X$ e cada morfismo em $\mathcal{J}$ para a identidade $1_X$ naquele objeto. Ou seja, essa composição induz um funtor $[C \cong Fun(\textbf{1}, \mathcal{C})] \to Fun(\mathcal{J}, \mathcal{C})$ denotado por $\Delta_t : \mathcal{C} \to Fun(\mathcal{J}, \mathcal{C}) = \mathcal{C}^{\mathcal{J}}$. No total, isso dá $\Delta : C \to \mathcal{C}^{\mathcal{J}}$ que leva cada objeto $X$ para o funtor constante em $X$ e cada morfismo $f : X \to Y$ à \emph{transformação natural constante}. Seja $f : X \to Y$ em $\mathcal{C}$, então existe uma transformação natural $\Delta (X) \xrightarrow{\Delta (f)} \Delta (Y)$ tal que:

\[\begin{tikzcd}
	{(\Delta X)(i)} && {(\Delta Y)(i)} && i \\
	\\
	{(\Delta X)(j)} && {(\Delta Y)(j)} && j
	\arrow["{(\Delta f)(i)}", from=1-1, to=1-3]
	\arrow["{(\Delta X)(e)}"', from=1-1, to=3-1]
	\arrow["{(\Delta Y)(e)}", from=1-3, to=3-3]
	\arrow["e"', from=1-5, to=3-5]
	\arrow["{(\Delta f)(j)}"', from=3-1, to=3-3]
\end{tikzcd}\]

comuta para cada aresta $e$ da categoria índice $\mathcal{J}$. Mas como $\Delta$ leva para o funtor constante, que leva os objetos neles prórios, então os objetos nesse diagrama se reduzem a:

\[\begin{tikzcd}
	X && Y \\
	\\
	X && Y
	\arrow["{(\Delta f)(i)}", from=1-1, to=1-3]
	\arrow["{1_X}"', from=1-1, to=3-1]
	\arrow["{1_Y}", from=1-3, to=3-3]
	\arrow["{(\Delta f)(j)}"', from=3-1, to=3-3]
\end{tikzcd}\]

que obviamente comuta.

Considerando um $\mathcal{J}$-diagrama qualquer $F : \mathcal{J} \to \mathcal{C}$ e para cada $X \in \mathcal{C}$, as setas (que são transformações naturais):

$$\Delta X \to F \quad\quad F \to \Delta X$$

Então uma seta em $\mathcal{C}^{\mathcal{J}}$ que corresponde a essas setas é somente uma transformação natural, ou seja uma familia de setas em $\mathcal{C}$,

$$(\Delta X)(i) \xrightarrow{\xi (i)} F(i) \quad\quad F(i) \xrightarrow{\xi (i)} (\Delta X)(i)$$

indexada pelos vários objetos ou nós em $\mathcal{J}$ e tais que:

\[\begin{tikzcd}
	{(\Delta X)(i)} && {F(i)} && i && {F(i)} && {(\Delta X)(i)} \\
	\\
	{(\Delta X)(j)} && {F(j)} && j && {F(j)} && {(\Delta X)(j)}
	\arrow["{\xi (i)}", from=1-1, to=1-3]
	\arrow["{(\Delta X)(e)}"', from=1-1, to=3-1]
	\arrow["{F(e)}", from=1-3, to=3-3]
	\arrow["e"', from=1-5, to=3-5]
	\arrow["{\xi (i)}", from=1-7, to=1-9]
	\arrow["{F(e)}"', from=1-7, to=3-7]
	\arrow["{(\Delta X)(e)}", from=1-9, to=3-9]
	\arrow["{\xi(j)}"', from=3-1, to=3-3]
	\arrow["{\xi (j)}"', from=3-7, to=3-9]
\end{tikzcd}\]

Ou, fazendo a mesma redução anterior:

\[\begin{tikzcd}
	X && {F(i)} && i && {F(i)} && X \\
	\\
	X && {F(j)} && j && {F(j)} && X
	\arrow["{\xi (i)}", from=1-1, to=1-3]
	\arrow["{id_X}"', from=1-1, to=3-1]
	\arrow["{F(e)}", from=1-3, to=3-3]
	\arrow["e"', from=1-5, to=3-5]
	\arrow["{\xi (i)}", from=1-7, to=1-9]
	\arrow["{F(e)}"', from=1-7, to=3-7]
	\arrow["{id_X}", from=1-9, to=3-9]
	\arrow["{\xi(j)}"', from=3-1, to=3-3]
	\arrow["{\xi (j)}"', from=3-7, to=3-9]
\end{tikzcd}\]

Mas isso se reduz aos seguintes triangulos comutativos:

\[\begin{tikzcd}
	&& {F(i)} && i && {F(i)} \\
	X &&&&&&&& X \\
	&& {F(j)} && j && {F(j)}
	\arrow["{F(e)}", from=1-3, to=3-3]
	\arrow["e"', from=1-5, to=3-5]
	\arrow["{F(e)}"', from=1-7, to=3-7]
	\arrow["{\xi (i)}", from=2-1, to=1-3]
	\arrow["{\xi (j)}"', from=2-1, to=3-3]
	\arrow["{\xi(i)}"', from=2-9, to=1-7]
	\arrow["{\xi (j)}", from=2-9, to=3-7]
\end{tikzcd}\]


As transformações naturais representadas pelos triangulos na esquerda dão \emph{soluções esquerdas} ao diagrama em $\mathcal{C}$, as vezes chamadas de \emph{Cones acima} do diagrama $F$ com vertice \emph{cume} em $X$. As transformações naturais representada pelos triangulos a direita dão as \emph{soluções direitas} para o diagrama, também chamadas de \emph{Cocones abaixo} do diagrama $F$ com \emph{nadir} (ponto mais baixo) em $X$.

Com isso, é possível definir a categoria de cones, onde um objeto na categoria de cones sobre $F$ será um cone acima de $F$, com algum cume, enquanto o morfismo de um cone $\xi : X \Rightarrow F$ para um cone $\mu : Z \Rightarrow F$ é um morfismo $f : X \to Z$ em $\mathcal{C}$ tal que para cada indice $j \in \mathcal{J}$, $\mu_j \circ f = \xi_j$, ou seja, um mapa entre cumes tal que cada perna do cone domínio é fatorado através da perna correspondente do cone codominio.

A categoria de cones pode ser vista como a categoria slice $\Delta / F$.

Usando essas noções, é possível definir o \emph{limite} de $F$ em termos do cone universal, onde um cone $\alpha : L \to F$ com vertice $L$ é universal em respeito a $F$ caso para cada cone $\Delta X \to F$ existe um mapa único $g : \Delta X \to L$ tal que o diagrama:

% \[\begin{tikzcd}
% 	X && L \\
% 	& {F(i)} \\
% 	& {F(j)}
% 	\arrow["g", dashed, from=1-1, to=1-3]
% 	\arrow["{\xi (i)}", from=1-1, to=2-2]
% 	\arrow["{\xi (j)}"', curve={height=18pt}, from=1-1, to=3-2]
% 	\arrow["{\alpha (i)}"', from=1-3, to=2-2]
% 	\arrow["{\alpha(j)}", curve={height=-18pt}, from=1-3, to=3-2]
% 	\arrow["{F(e)}"', from=2-2, to=3-2]
% \end{tikzcd}\]

Comuta. Em uma definição formal:

\begin{definition}[\cite{awodey2010}]
    Um \emph{limite} para o diagrama $F : \mathcal{J} \to \mathcal{C}$ é um objeto terminal $\text{lim} F$ na categoria de cones em $F$, denotada por $\textbf{Cone}(F)$. Se $\mathcal{J}$ for finito, o limite é chamado de \emph{limite finito}
\end{definition}

De forma dual, na categoria dos cocones $\textbf{CoCones}(F)$ o cocone universal surge como \emph{colimite} do diagrama $F$, denotado por $\text{colim} F$:

\begin{definition}[\cite{rosiak2022}]
    O \emph{colimite} do diagrama $F : \mathcal{J} \to \mathcal{C}$ é um objeto $\text{colim} F$ em $\mathcal{C}$ junto com uma transformação natural $\epsilon : F \Rightarrow \text{colim} F$ que satisfaz que para qualquer objeto $X$ e qualquer transformação natural $\beta : F \Rightarrow X$ existe um morfismo único $h : \text{colim} F \to X$ tal que $\beta = h \circ \epsilon$
\end{definition}

Exemplos de limites:

Em \textbf{Set}, o limite do diagrama discreto consistindo de conjuntos $X_1, X_2, \dots$, é o \emph{Produto cartesiano} $\Pi_{i \in I} x_i$, onde essa construção vem com mapas projetivos $\pi_i : \Pi_{i \in I} X_i \to X_i$ para cada fator.

Seja o diagrama na forma:

\[\begin{tikzcd}
	&& \bullet \\
	\\
	\bullet && \bullet
	\arrow[from=1-3, to=3-3]
	\arrow[from=3-1, to=3-3]
\end{tikzcd}\]

em uma categoria $\mathcal{C}$, o limite de um diagrama de tal formato é chamado de \emph{pullback} (ou \emph{produto fibrado}), consistindo de um objeto junto com morfismos que satisfazem essa propriedade universal. De forma formal:

\begin{definition}\cite{awodey2010}
    Em uma categoria $\mathcal{C}$, um \emph{pullback} de setas $f, g$ com $cod (f) = cod (g)$
    \[\begin{tikzcd}
        && B \\
        \\
        A && C
        \arrow["f"', from=1-3, to=3-3]
        \arrow["f", from=3-1, to=3-3]
    \end{tikzcd}\]

    consiste de setas

    \[\begin{tikzcd}
        P && B \\
        \\
        A
        \arrow["{\pi_1}"', from=1-1, to=1-3]
        \arrow["{\pi_2}"', from=1-1, to=3-1]
    \end{tikzcd}\]

    tais que $f \pi_1 = g \pi_2$, ou seja o diagrama

    \[\begin{tikzcd}
        P && B \\
        \\
        A && C
        \arrow["{\pi_1}"', from=1-1, to=1-3]
        \arrow["{\pi_2}"', from=1-1, to=3-1]
        \arrow["g", from=1-3, to=3-3]
        \arrow["f"', from=3-1, to=3-3]
    \end{tikzcd}\]

    comuta.
    
    E também $\pi_1$ e $\pi_2$ são universais com essa propriedade. Ou seja sendo quaisquer $z_1 : Z \to A$ e $z_2 : Z \to B$, com $fz_1 = gz_2$, então existe um morfismo único $$u : Z \to P$$ com $z_1 = \pi_1 u$ e $z_2 = \pi_2 u$. O diagrama equivalente é:

    \[\begin{tikzcd}
        Z \\
        & P && B \\
        \\
        & A && C
        \arrow[dashed, from=1-1, to=2-2]
        \arrow["{z_2}", from=1-1, to=2-4]
        \arrow["{z_1}"', from=1-1, to=4-2]
        \arrow["{\pi_2}", from=2-2, to=2-4]
        \arrow["{\pi_1}"', from=2-2, to=4-2]
        \arrow["g", from=2-4, to=4-4]
        \arrow["f"', from=4-2, to=4-4]
    \end{tikzcd}\]
\end{definition}

Considerando um poset como uma categoria, um diagrama pullback:

% https://q.uiver.app/#q=WzAsMyxbMCwxLCJwIl0sWzEsMCwicSJdLFsxLDEsInIiXSxbMSwyLCJcXGxlcSJdLFswLDIsIlxcbGVxIiwyXV0=
\[\begin{tikzcd}
	& q \\
	p & r
	\arrow["\leq", from=1-2, to=2-2]
	\arrow["\leq"', from=2-1, to=2-2]
\end{tikzcd}\]

será dado por um elemento $l$, com $l \leq p$ e $l \leq q$, tais que para qualquer elemento $s$ que também $s \leq p$ e $s \leq q$, tem-se que $s \leq l$:

% https://q.uiver.app/#q=WzAsNSxbNCwyLCJxIl0sWzIsNCwicCJdLFs0LDQsInIiXSxbMiwyLCJsIl0sWzAsMCwicyJdLFsxLDIsIlxcbGVxIiwyXSxbMCwyLCJcXGxlcSJdLFszLDAsIlxcbGVxIl0sWzMsMSwiXFxsZXEiLDJdLFs0LDEsIlxcbGVxIiwyXSxbNCwwLCJcXGxlcSJdLFs0LDMsIlxcbGVxIiwyLHsic3R5bGUiOnsiYm9keSI6eyJuYW1lIjoiZGFzaGVkIn19fV1d
\[\begin{tikzcd}
	s \\
	\\
	&& l && q \\
	\\
	&& p && r
	\arrow["\leq"', dashed, from=1-1, to=3-3]
	\arrow["\leq", from=1-1, to=3-5]
	\arrow["\leq"', from=1-1, to=5-3]
	\arrow["\leq", from=3-3, to=3-5]
	\arrow["\leq"', from=3-3, to=5-3]
	\arrow["\leq", from=3-5, to=5-5]
	\arrow["\leq"', from=5-3, to=5-5]
\end{tikzcd}\]

Ou seja, $l$ é a \emph{maior cota inferior} de $p$ e $q$

Para \textbf{Sets} existem várias formas de construir pullbacks (\cite{rosiak2022}):

\begin{itemize}
    \item Seja $Z = \{\ast\}$ um conjunto unitário. ENtão como $\{\ast\}$ é o objeto terminal em \textbf{Sets}, tanto $X \xrightarrow{f} \{\ast\}$ e $Y \xrightarrow{g} \{\ast\}$ são funções únicas que levam tudo para $\{\ast\}$. O pullback de tal diagrama seria todos os pares $(x, y)$ tais que tanto $x$ e $y$ são enviados para $\{\ast\}$ pelas funções únicas $f$ e $g$, mas como não existem pares que satisfaçam esse requisito, é possível recuperar \emph{todos} os pares $(x, y)$ fazendo o pullback um diagrama que possui o conjunto inteiro $X \times Y$, chamado de produto cartesiano binário.
    \item Agora seja $Y = \{\ast\}$, enquanto $Z$ é qualquer conjunto. Uma função $\{\ast\} \xrightarrow{g} Z$ só pega um elemento $z \in Z$. Então, para qualquer função $X \xrightarrow{f} Z$, o pullback será o subconjunto de elementos em $X$ que são enviados para $z$ através de $f$, recuperando a \emph{pre-imagem} (ou \emph{fibra}) de $f$ em $g$.
    \item Agora sejam $X$ e $Y$ subconjuntos de $Z$, fazendo que $f$ e $g$ sejam inclusões, da forma:
    % https://q.uiver.app/#q=WzAsMyxbMCwyLCJYIl0sWzIsMiwiWiJdLFsyLDAsIlkiXSxbMCwxLCJmIiwyLHsic3R5bGUiOnsidGFpbCI6eyJuYW1lIjoiaG9vayIsInNpZGUiOiJ0b3AifX19XSxbMiwxLCJnIiwwLHsic3R5bGUiOnsidGFpbCI6eyJuYW1lIjoiaG9vayIsInNpZGUiOiJ0b3AifX19XV0=
\[\begin{tikzcd}
	&& Y \\
	\\
	X && Z
	\arrow["g", hook, from=1-3, to=3-3]
	\arrow["f"', hook, from=3-1, to=3-3]
\end{tikzcd}\]
Nesse caso, o pullback vai consistir em pares $(x, y)$ tais que $x$ e $y$ são \emph{iguais} na inclusão em $Z$. Ou seja, o pullback consiste em elementos $x = y$ de $X$ que também estão em $Y$. Isso é a construção da \emph{interseção} $X \cap Y$ 
\end{itemize}

Seja um diagrama da forma 

\[\begin{tikzcd}
	\bullet && \bullet && \bullet & \dots
	\arrow[from=1-3, to=1-1]
	\arrow[from=1-5, to=1-3]
\end{tikzcd}\]

Em uma categoria $\mathcal{C}$, o limite de tal diagrama é chamado de \emph{limite inverso}. Esse limite possui um objeto junto com os morfismos daquele objeto para cada $\bullet$ tal que todos os triangulos resultantes comutem e a propriedade universal do limite seja satisfeita.

Sejam $X_1, X_2, \dots$ objetos em $\mathcal{C}$, então o limite inverso, denotado $\underset{\leftarrow}{lim} X_i$ do diagrama 

\[\begin{tikzcd}
	{X_1} && {X_2} && {X_3} & \dots
	\arrow["{f_1}", from=1-3, to=1-1]
	\arrow["{f_2}", from=1-5, to=1-3]
\end{tikzcd}\]

seria um objeto que mapeia em cada $X_i$ na forma:

\[\begin{tikzcd}
	&& {\underset{\leftarrow}{lim} X_i} \\
	\\
	{X_1} && {X_2} && {X_3} & \dots
	\arrow[from=1-3, to=3-1]
	\arrow[from=1-3, to=3-3]
	\arrow[from=1-3, to=3-5]
	\arrow["{f_1}", from=3-3, to=3-1]
	\arrow["{f_2}", from=3-5, to=3-3]
\end{tikzcd}\]

Em \textbf{Set}, esse seria um subconjunto $\underset{\leftarrow}{lim} X_i$ do produto $\prod_{i \in I} X_i$ contendo todas as sequências $(x_1, x_2, x_3, \dots)
$ onde o $i$-ésimo fator é tal que $f_i (x_{i+1}) = x_i$

Exemplo (\cite{rosiak2022}): Seja um diagrama indexado por uma categoria consistindo de dois objetos e dois morfismos paralelos não identidade,

\[\begin{tikzcd}
	\bullet && \bullet
	\arrow[shift right, from=1-1, to=1-3]
	\arrow[shift left, from=1-1, to=1-3]
\end{tikzcd}\]

Um diagrama em $\mathcal{C}$ de tal forma é um par de morfismos

\[\begin{tikzcd}
	X && Y
	\arrow["g"', shift right, from=1-1, to=1-3]
	\arrow["f", shift left, from=1-1, to=1-3]
\end{tikzcd}\]

em $\mathcal{C}$. Um cone com cume $C$ consiste em de pares de morfismos $h : C \to X$ e $i : C \to Y$ tais que $f \circ h = i$ e $g \circ h = i$ junto com $f \circ h = g \circ h$. Ou seja, um cone acima desse par é um morfismo $h : C \to X$ tal que $f \circ h = g \circ h$. 

A partir disso, é possível definir um objeto $E$ junto com $e : E \to X$, chamado de \emph{equalizador} de $f$ e $g$, como a seta universal com a mesma propriedade, ou seja, $f \circ e = g \circ e$. Essa propriedade universal aponta que dado qualquer $h : C \to X$ tal que $f \circ h = g \circ h$ existe uma seta única $k : C \to E$ que fatora o morfismo $h$ através de $e$ tal que $e \circ k = h$ como no diagrama:

\[\begin{tikzcd}
	C \\
	\\
	E && X && Y
	\arrow["k"', dashed, from=1-1, to=3-1]
	\arrow["h", from=1-1, to=3-3]
	\arrow["e", from=3-1, to=3-3]
	\arrow["g"', shift right, from=3-3, to=3-5]
	\arrow["f", shift left, from=3-3, to=3-5]
\end{tikzcd}\]

Em \textbf{Set}, o equalizador de $f$ e $g$ é o subconjunto de elementos de $X$ para os quais as duas funções coincidem: $$E = Eq(f, g) := \{x \in X | f(x) = g(x)\}$$

A seguinte definição mostra quando é possível "cancelar" setas em um lado:

\begin{definition}[monomorfismo, \cite{rosiak2022}]
    Um morfismo $i : B \to C$ em uma categoria é chamado de \emph{monomorfismo} (ou morfismo \emph{mônico}) se para qualquer $A$ com morfismos paralelos $f$ e $g$ tais que:

    \[\begin{tikzcd}
        A && B && C
        \arrow["g"', shift right, from=1-1, to=1-3]
        \arrow["f", shift left, from=1-1, to=1-3]
        \arrow["i"', from=1-3, to=1-5]
    \end{tikzcd}\]

    $i \circ f = i \circ g$ implica que $f = g$
\end{definition}

monomorfismos generalizam a noção de funções injetoras em \textbf{Set}.

\begin{proposition}[\cite{awodey2010}]
    Em qualquer categoria, se $e : E \to A$ é um equalizador de um par de setas, então $e$ é mônico
\end{proposition}

\emph{Prova}: Considere o diagrama:

\[\begin{tikzcd}
	E && A && B \\
	\\
	Z
	\arrow["e"', from=1-1, to=1-3]
	\arrow["g"', shift right, from=1-3, to=1-5]
	\arrow["f", shift left, from=1-3, to=1-5]
	\arrow["x", shift left, from=3-1, to=1-1]
	\arrow["y"', shift right, from=3-1, to=1-1]
	\arrow["z", from=3-1, to=1-3]
\end{tikzcd}\]

onde $e$ é equalizador de $f$ e $g$. Suponha que $ex = ey$, o que quer se mostrar é que $x = y$. Para isso, vamos usar a comutatividade do diagrama triangular. Seja $z = ex = ey$, então $fz = fex = gex = gz$, então existe um morfismo único $u : Z \to E$ tal que $eu = z$. Mas $ex = z$ e $ey = z$, então se segue que $x = u = y$. $\Box $

\begin{definition}[\cite{awodey2010}]
    Para mapas $r$ e $s$, em qualquer categoria, $r$ é chamado de \emph{retração} de $s$ caso $r \circ s$ for um mapeamento identidade. Nessa situação, $s$ é chamado de \emph{seção} de $r$
\end{definition}

Exemplos de colimites

Em \textbf{Set}, o colimite de um diagrama discreto consistindo de conjuntos $X_1, X_2, \dots$ é a \emph{união disjunta} $\sqcup_{i \in I} X_i$, construida a partir de funções injetivas de cada $X_i$ no conjunto coproduto $X =  \sqcup_{i \in I} X_i$. Se os conjuntos são disjuntos par a par, a união disjunta se torna a união padrão $\cup$.

Em um poset, o colimite de um diagrama discreto é o seu \emph{supremo} (ou \emph{menor cota superior}) $\bigvee_{i \in I} p_i$.

O dual de limites inversos são \emph{limites diretos}, que são o colimite do diagrama indexado pela categorial ordinal $\omega$. Ou seja, para o diagrama

\[\begin{tikzcd}
	{X_1} & {X_2} & {X_3} & {X_4} & \dots
	\arrow[from=1-1, to=1-2]
	\arrow[from=1-2, to=1-3]
	\arrow[from=1-3, to=1-4]
	\arrow[from=1-4, to=1-5]
\end{tikzcd}\]

seu colimite é o limite direto $\underset{\to}{lim} X_n$, que define o diagrama de forma $\omega + 1$:

\[\begin{tikzcd}
	{X_1} & {X_2} & {X_3} & {X_4} & \dots \\
	\\
	&&& {\underset{\to}{lim} X_n}
	\arrow[from=1-1, to=1-2]
	\arrow[from=1-1, to=3-4]
	\arrow[from=1-2, to=1-3]
	\arrow[from=1-2, to=3-4]
	\arrow[from=1-3, to=1-4]
	\arrow[from=1-3, to=3-4]
	\arrow[from=1-4, to=1-5]
	\arrow[from=1-4, to=3-4]
	\arrow[from=1-5, to=3-4]
\end{tikzcd}\]

O colimite de uma sequência de conjuntos formada por inclusões:

\[\begin{tikzcd}
	{X_1} & {X_2} & {X_3} & {X_4} & \dots
	\arrow[hook, from=1-1, to=1-2]
	\arrow[hook, from=1-2, to=1-3]
	\arrow[hook, from=1-3, to=1-4]
	\arrow[hook, from=1-4, to=1-5]
\end{tikzcd}\]

é a sua união $\bigcup_{n \geq 0} X_n$

\begin{definition}[Coequalizador, \cite{awodey2010}]
    Sejam duas setas paralelas $f, g : A \to B$ em uma categoria $\mathcal{C}$, um \emph{coequalizador} consiste de $Q$ e $q : B \to Q$ universais com a propriedade $qf = qg$ como no diagrama:
    \[\begin{tikzcd}
        A && B && Q \\
        \\
        &&&& Z
        \arrow["f", shift left, from=1-1, to=1-3]
        \arrow["g"', shift right, from=1-1, to=1-3]
        \arrow["q", from=1-3, to=1-5]
        \arrow["z"', from=1-3, to=3-5]
        \arrow["u", dashed, from=1-5, to=3-5]
    \end{tikzcd}\]

    Ou seja, dado qualquer objeto $Z$ e $z : B \to Z$, se $zf = zg$, então existe um morfismo único $u : Q \to Z$ tal que $uq = z$
\end{definition}

Em \textbf{Set}, o coequalizador de duas funções $f, g : X \to Y$ é o \emph{quociente} de $Y$ pela menor relação de equivalência $\sim$ tal que para todo $x \in X$, $f(x) \sim g(x)$

\begin{definition}[\cite{rosiak2022}]
    $f: X \to Y$ é dito \emph{epimorfismo} (ou somente \emph{epi}) se para todo $B$ e morfismos $h, h' : Y \to B$,
    \[\begin{tikzcd}
        X && Y && B
        \arrow["f", from=1-1, to=1-3]
        \arrow["h", shift left, from=1-3, to=1-5]
        \arrow["{h'}"', shift right, from=1-3, to=1-5]
    \end{tikzcd}\]

    $h \circ f = h' \circ f$ implica que $h = h'$
\end{definition}

Epimorfismos podem ser vistos como generalizações das funções sobrejetoras em \textbf{Set}.

\begin{proposition}[\cite{awodey2010}]
    se $q : B \to Q$ é um coequalizador de um par de setas, então $q$ é epi.
\end{proposition}


\begin{definition}[\cite{rosiak2022}]
    Seja $\mathcal{C}$ uma categoria e seja $F : \mathcal{C} \to \textbf{Set}$ um funtor covariante. Então a \emph{categoria de elementos} de $F$ denotada $\int_{\mathcal{C}} F$ (ou somente $\int F$ se o contexto for claro) é definida:
    \begin{itemize}
        \item $Ob(\int F) = \{(c, x) | c \in \mathcal{C}, x \in F(c)\}$
        \item $Hom_{\int F}((c,x), (c', x')) = \{f : c \to c' | F(f)(x) = x'\}$
    \end{itemize} 

    De forma dual, para o caso contravariante: para $F : \mathcal{C}^{op} \to \textbf{Set}$, a \emph{categoria de elementos} de $F$ denotada $\int_{\mathcal{C}^op} F$ (ou somente $\int F$) é definida por:
    \begin{itemize}
        \item $Ob(\int F) = \{(c, x) | c \in \mathcal{C}, x \in F(c)\}$
        \item $Hom_{\int F}((c,x), (c', x')) = \{f : c \to c' | F(f)(x') = x\}$
    \end{itemize} 
    Associado com essa construção estão os funtores $\pi_F : \int F \to \mathcal{C}$, chamados de \emph{funtores de projeção}, que mandam cada objeto $(c, x) \in Ob(\int F)$ para o objeto $c \in Ob(\mathcal{C})$ ou $Ob(\mathcal{C}^{op})$ e cada morfismo $f : (c, x) \to (c', x')$ para o morfismo $f : c \to c'$.
\end{definition}

\begin{definition}
    Para qualquer classe de diagramas $K : \mathcal{J} \to \mathcal{C}$ em $\mathcal{C}$, um funtor $F : \mathcal{D} \to \mathcal{C}$ \emph{preserva limites} se para qualquer diagrama $K$ e cone limite sobre $K$, a imagem desse cone sob a ação do funtor definite um cone limite sobre o diagrama composto $F \circ K : \mathcal{J} \to \mathcal{D}$.
\end{definition}


\begin{definition}[\cite{rosiak2022}]
    Um funtor é dito (co)\emph{continuo} se ele preserva dos os (co)\emph{limites} pequenos
\end{definition}

\subsection{Categorias Cartesianas Fechadas}

\subsubsection{Exponenciais}

Primeiro, é necessário uma digreção dentro da categoria dos conjuntos.

Seja a função entre conjuntos $$f(x, y) : A \times B \to C$$ escrita usando variáveis $x \in A$ e $y \in B$.

Se $a \in A$ for mantido constante, então tem-se a função $$f(a, y) : B \to C$$ e então o elemento $$f(a, y) \in C^B$$ do conjunto $C^B$ de todas as funções de $B$ para $C$. 

Se $a$ for variado, então pode-se criar outra função $$\bar{f} : A \to C^B$$ que leva como parâmetro $a$ para a função $f_a(y) : B \to C$.

A relação entre essas funções pode ser determinada pela seguinte equação:

$$\bar{f}(a)(b) = f(a, b)$$

Ou seja, existe um isomorfismo entre conjuntos:

$$Sets(A \times B, C) \cong Sets(A, C^B)$$

Ou seja, existe uma correspondência bijetiva entre essas funções mediada por uma operação de \emph{avaliação} (\emph{evaluation}):

$$eval : C^B \times B \to C$$

dada por:

$$eval(g, b) = g(b)$$

É possível definir exponênciais em qualquer categoria na seguinte forma:

\begin{definition}[\cite{awodey2010}]
    Seja a categoria $\mathcal{C}$ que possui produtos binários. Um \emph{exponencial} de objetos $B$ e $C$ de $\mathcal{C}$ consiste em um objeto $$C^B$$ e uma seta $$\epsilon : C^B \times B \to C$$ tal que, para todo objeto $Z$ e seta $f : Z \times B \to C$ existe uma seta única $$\bar{f} : Z \times C^B$$ tal que $$\epsilon \circ (\bar{f} \times 1_B) = f$$ como no diagrama:
    \[\begin{tikzcd}
        {C^B} && {C^B \times B} && C \\
        \\
        Z && {Z \times B}
        \arrow["\epsilon", from=1-3, to=1-5]
        \arrow["{\bar{f}}", from=3-1, to=1-1]
        \arrow["{\bar{f} \times 1_B}", from=3-3, to=1-3]
        \arrow["f"', from=3-3, to=1-5]
    \end{tikzcd}\]
\end{definition}

\subsubsection{Categorias Cartesianas Fechadas}

\begin{definition}[\cite{awodey2010}]
    Uma categoria $\mathcal{C}$ é dita ter \emph{todos os produtos finitos} se ele possui um objeto terminal e todos os produtos binários (e também produtos de qualquer cardinalidade finita). A categoria $\mathcal{C}$ possui \emph{todos os produtos pequenos} se todo conjunto de objetos em $\mathcal{C}$ possui produtos 
\end{definition}

Com isso, é possível definir categorias cartesianas fechadas da seguinte forma:

\begin{definition}[\cite{awodey2010}]
    Uma categoria é chamada \emph{cartesiana fechada} se possui todos os produtos finitos e exponenciais
\end{definition}




\end{document}


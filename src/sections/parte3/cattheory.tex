\documentclass[../main.tex]{subfiles}

\begin{document}

\section{Introdução à Teoria das Categorias}

A teoria das categorias é uma área de matemática que relaciona diversas áreas, como por exemplo, Teoria dos Grupos, Teoria dos Anéis, Topologia, Teoria dos Grafos, etc. Cada uma dessas teorias tem em comum a definição de seus objetos (Grupos, Anéis, Espaços topológicos, grafos) e formas de relacionar esses objetos (Homomorfismos de grupos, homomorfismos de aneis, homeomorfismos, homomorfismos entre grafos). 

\subsection{Categorias}

Para estudar categorias, primeiro é necessário defini-las:

\begin{definition}[Categoria, \cite{awodey2010}]
    Uma \emph{categoria} $\mathbf{C}$ consiste em:
    \begin{itemize}
        \item \emph{Objetos}: $A, B, C, \dots$
        \item \emph{Setas} (Morfismos): $f, g, h, \dots$
        \item Para cada seta $f$ existem objetos: $$dom(f), cod(f)$$ chamados de \emph{domínio} e \emph{contradomínio} de $f$. A escrita $$f : A \to B$$ indica que $A = dom(f)$ e $B = cod(f)$
        \item Sejam setas $f : A \to B$ e $g : B \to C$ com: $$cod(f) = dom(g)$$ existe uma seta $g \circ f : A \to C$ chamada de \emph{composição} de $f$ com $g$
        \item Para cada objeto $A$ existe uma seta $$1_A : A \to A$$ chamada de \emph{seta identidade} de $A$
    \end{itemize}
    Esses dados precisam satisfazer os seguintes axiomas:
    \begin{itemize}
        \item (Associatividade) Sejam $f : A \to B$, $g : B \to C$ e $h : C \to D$ setas, então: $$h \circ (g \circ f) = (h \circ g ) \circ f$$
        \item (Identidade) Seja $f : A \to B$ uma seta, então $$f \circ 1_A = f = 1_B \circ f$$
    \end{itemize}
\end{definition}

Para quaisquer objetos $A$ e $B$ em uma categoria $C$, a coleção de setas de $A$ para $B$ é escrito $Hom_C(A, B)$

Alguns exemplos de categorias são:

\begin{enumerate}
    \item A categoria \textbf{Set} que possui conjuntos como objetos e funções como morfismos. 
    \item Os conjuntos ordenados descritos na Definição 1.31 também podem formar uma categoria junto com os mapeamentos monótonos descritos na Definição 1.32, chamada de \textbf{Pos}
    \item Um monóide é um conjunto $M$ equipado com uma operação binária $\cdot : M \times M \to M$ e um elemento unitário $e \in M$ tal que para todo $x, y, z \in M$: $$x \cdot (y \cdot z) = (x \cdot y) \cdot z$$ e $$e \cdot x = x = x \cdot e$$. Por exemplo, o conjunto dos naturais $\mathbb{N}$, junto à operação de soma usual $+ : \mathbb{N} \times \mathbb{N} \to \mathbb{N}$, pode ser considerado um monoide, com o $0$ como elemento unitário. \\
    Dois monóides $(M, \cdot)$ e $(N, \star)$ podem ser relacionados através de um \emph{homomorfismo} $\phi : M \to N$ tal que $$\phi (x \cdot y) = \phi(x) \star \phi(y)$$ e $$\phi(e_M) = e_N$$
    A categoria que possui monóides como objetos e homeomorfismos como morfismos é denominada de \textbf{Mon}
    \item Um grupo $G$ é um monóide onde para todo $a \in G$ existe um elemento $b \in G$ tal que $a \cdot b = e$. $b$ é chamado de \emph{inverso} de $a$ e é escrito como $a^{-1}$. Um homomorfismo $\phi$ entre dois grupos $(G, \cdot)$ e $(H, \star)$ obedece as duas condições para homomorfismos entre monóides mais a seguinte: $$\phi(a^{-1}) = \phi(a)^{-1}$$ A categoria que possui monóides como objetos e homomorfismos como morfismos é denominada de \textbf{Grp}
    \item (\cite{riehl2017}) Um grupo $G$ (e também um monóide) define uma categoria $BG$ com um único objeto. Os elementos do grupo são seus morfismos e a composição é dada por $\cdot$. O elemento unitário $e \in G$ age como o morfismo identidade para o objeto único dessa categoria. \\ Por exemplo, para $(\mathbb{Z}, +)$, $e = 0$ e será representado por $0 : \mathbb{Z} \to \mathbb{Z}$. Sendo $ 1 : \mathbb{Z} \to \mathbb{Z}$ e $2 : \mathbb{Z} \to \mathbb{Z}$, então a composição $1 \circ 2$ é em $(\mathbb{Z}, +)$ equivalente a $1 + 2$ e $1 \circ 2 = 3$.
\end{enumerate}

\begin{definition}[Isomorfismos, \cite{awodey2010}]
    Em qualquer categoria $C$, um morfismo $f : A \to B$ é chamado de \emph{isomorfismo} se existe um morfismo $g : B \to A$ em $C$ tal que $$g \circ f = 1_A \text{ e } f \circ g = 1_B$$
\end{definition}

$g$ é chamado de inverso de $f$ e, por ser único, pode ser denotado por $f^{-1}$. Os objetos $A$ e $B$ são ditos \emph{isomórficos} e denotados por $A \cong B$

Exemplos:

\begin{enumerate}
    \item Os isomorfismos em \textbf{Set} são bijeções
    \item Os isomorfismos em \textbf{Grp} são os homomorfismos bijetivos
\end{enumerate}

\begin{definition}[Categorias pequenas, \cite{awodey2010}]
    Uma categoria $C$ é chamada de \emph{pequena} se a coleção $C_0$ de objetos em $C$ e a coleção $C_1$ de morfismos em $C$ são conjuntos. Caso contrário, $C$ é chamada de \emph{grande}
\end{definition}

Todas as categorias finitas são pequenas, assim como a categoria \textbf{$Sets_{fin}$} de conjuntos finitos. Já a categoria \textbf{Sets} é grande (Pois caso a coleção de seus objetos fosse um conjunto, isso geraria o paradoxo de Russell)

\begin{definition}[Categoria localmente pequena, \cite{awodey2010}]
    Uma categoria $C$ é chamada de \emph{localmente pequena} se para quaisquer objetos $X$ e $Y$ em $C$, a coleção de morfismos $Hom_C(X, Y) = \{f \in C_1 | f : X \to Y\}$ é um \emph{conjunto} (Chamado de \emph{hom-set})
\end{definition}

\subsection{Categorias novas das antigas}

Dada a definição de categorias, é interessante analisar o que pode ser feito com uma categoria e como gerar novas categorias de categorias antigas

\begin{definition}[Categoria oposta, \cite{awodey2010}]
    A categoria \emph{oposta} (ou "dual") $C^{op}$ de uma categoria $C$ possui os mesmos objetos que $C$, mas para cada morfismos $f : A \to B$ em $C$ existe um morfismo $f : B \to A$ em $C^{op}$
\end{definition}

A categoria oposta inverte todos os morfismos da categoria que parte. Então seja $f^{op}$ o morfismo invertido, a composição na categoria oposta se torna: $f^{op} \circ g^{op} = (g \circ f)^{op}$

É interessante perceber que cada resultado na Teoria das Categorias terá um resultado dual ganho "de graça" ao fazer esse resultado nas categorias duais.

Também é possível ver que $(C^{op})^{op} = C$

\begin{definition}[Categoria de setas, \cite{rosiak2022}]
    Seja uma categoria $C$, definimos a \emph{categoria de setas} de $C$, denotada por $C^{\to}$, tendo:
    \begin{itemize}
        \item Objetos: morfismos $A \xrightarrow{f} B$ de $C$
        \item Morfismos: a partir de um objeto de $C^{\to}$ $A \xrightarrow{f} B$ para outro $A' \xrightarrow{f'} B'$ um morfismo é um par $\langle A \xrightarrow{f} B, A' \xrightarrow{f'} B' \rangle$ de morfismos de $C$ fazendo o diagrama
        \[\begin{tikzcd}
            A && {A'} \\
            \\
            B && {B'}
            \arrow["h", from=1-1, to=1-3]
            \arrow["f"', from=1-1, to=3-1]
            \arrow["{f'}", from=1-3, to=3-3]
            \arrow["k"', from=3-1, to=3-3]
        \end{tikzcd}\]
        comutar. Ou seja, $k \circ f = f' \circ h$ em $C$
    \end{itemize}

    A composição das setas é feita ao colocar quadrados comutativos lado a lado da seguinte forma:
    \[\begin{tikzcd}
        A && {A'} && {A''} \\
        \\
        B && {B'} && {B''}
        \arrow["h", from=1-1, to=1-3]
        \arrow["f"', from=1-1, to=3-1]
        \arrow["l", from=1-3, to=1-5]
        \arrow["{f'}", from=1-3, to=3-3]
        \arrow["{f''}", from=1-5, to=3-5]
        \arrow["k"', from=3-1, to=3-3]
        \arrow["m", from=3-3, to=3-5]
    \end{tikzcd}\]

    tal que $\langle l, m \rangle \circ \langle h, k \rangle = \langle l \circ h, m \circ k \rangle$
    \\
    A identidade de um objeto $A \xrightarrow{f} B$ é dado pelo par $\langle id_A, id_B \rangle$

\end{definition}

Outro tipo de categoria de interesse é a categoria slice:

\begin{definition}[Categoria Slice, \cite{awodey2010}]
    A categoria slice $\textbf{C}/C$ de uma categoria $\textbf{C}$ sobre um objeto $C \in \textbf{C}$ possui:
    \begin{itemize}
        \item Objetos: todas as setas $f \in \textbf{C}$ tal que $cod(f) = C$
        \item Morfismos: $g$ de $f : X \to C$ e $f' : X' \to C$ é uma seta $g : X \to X'$ em $\textbf{C}$ tal que $f' \circ g = f$ como no diagrama:
        \[\begin{tikzcd}
            X && {X'} \\
            & C
            \arrow["g", from=1-1, to=1-3]
            \arrow["f"', from=1-1, to=2-2]
            \arrow["{f'}", from=1-3, to=2-2]
        \end{tikzcd}\]
    \end{itemize}
    A composição desses morfismos é basicamente a junção de desses triangulos \\

    Também é possível definir a categoria $(C / \textbf{C})$ chamada de categoria de co-slice, onde os objetos são setas $f$ de $\textbf{C}$ tal que $dom(f) = C$ e uma seta entre $f : C \to X$ e $f' : C \to X'$ é uma seta $h : X \to X'$ tal que $h \circ f = f'$ como no diagrama:
    \[\begin{tikzcd}
        & C \\
        X && {X'}
        \arrow["f", from=2-1, to=1-2]
        \arrow["g", from=2-1, to=2-3]
        \arrow["{f'}"', from=2-3, to=1-2]
    \end{tikzcd}\]
\end{definition}

Também é possível definir a noção de subcategoria:

\begin{definition}[Subcategoria, \cite{rosiak2022}]
    Uma categoria $\textbf{D}$ dita \emph{subcategoria} de $\textbf{C}$ é obtida restringindo a coleção de objetos de $\textbf{C}$ para uma subcoleção (Ou seja, todo $\textbf{D}$-objeto é um $\textbf{C}$-objeto) e a coleção de morfismos é obtida restringindo a coleção de morfismos de $\textbf{C}$ onde:
    \begin{itemize}
        \item Se o morfismo $f : A \to B$ está em $\textbf{D}$, então $A$ e $B$ estão em $\textbf{D}$
        \item Se $A$ está em $\textbf{D}$, então também está o morfismo identidade $id_A$
        \item Se $f : A \to B$ e $g : B \to C$ estão em $\textbf{D}$, então $g \circ f : A \to C$ também está
    \end{itemize}
\end{definition}

e também:

\begin{definition}[Subcategoria cheia, \cite{rosiak2022}]
    Seja $\textbf{D}$ uma subcategoria de $\textbf{C}$. ENtão $\textbf{D}$ é uma \emph{subcategoria cheia} de \textbf{C} quando \textbf{C} não possui setas $A \to B$ além dos que já existem em $\textbf{D}$. Ou seja para quaisquer objetos $A$ e $B$ em $\textbf{D}$, $\textbf{C}$: $$Hom_{\textbf{D}}(A, B) = Hom_{\textbf{C}}(A, B)$$
\end{definition}

Exemplo:

\begin{itemize}
    \item A categoria \textbf{FinSet} de conjuntos finitos é uma subcategoria de \textbf{Set}.
    \item Um grupo $(G, \cdot)$ é dito \emph{abeliano}, ou comutativo, caso para quaisquer dois elementos $a, b \in G$, $a \cdot b = b \cdot a$. A categoria de grupos abelianos \textbf{Ab} é uma subcategoria (cheia) de \textbf{Grp}
\end{itemize}

\subsection{Funtores}

Sendo categorias estruturas que se iniciam com objetos e morfismos entre esses objetos, é natural se perguntar se existem morfismos entre categorias. Esses morfismos deveriam também manter a estrutura entre categorias, como por exemplo os morfismos entre grupos mantém a estrutura do grupo, mesmo que mudando-se os objetos e os morfismos internos à categoria. Esse tipo de morfismo entre categorias é chamado de \emph{funtor} e definido da seguinte forma:

\begin{definition}[Funtor, \cite{awodey2010}]
    Um funtor $F : \mathcal{C} \to \mathcal{D}$ entre categorias $\mathcal{C}$ e $\mathcal{D}$ é um mapeamento de objetos em objetos e setas em setas tal que:
    \begin{enumerate}
        \item $F(f : A \to B) = F(f) : F(A) \to F(B)$
        \item $F(g \circ f) = F(g) \circ F(f)$
        \item $F(1_A) = 1_{F(A)}$
    \end{enumerate}
\end{definition}

A primeira parte define que para cada objeto $A$ em $\mathcal{C}$, existe um objeto correspondente $F(A)$ em $\mathcal{D}$. Também define que Funtores preservam os domínios e codomínios de cada morfismo.

\[\begin{tikzcd}
	A && B &&& {F(A)} && {F(B)} \\
	\\
	&& C &&&&& {F(C)} \\
	& {\mathcal{C}} &&&&& {\mathcal{D}}
	\arrow["f", from=1-1, to=1-3]
	\arrow["{g \circ f}"', from=1-1, to=3-3]
	\arrow["g", from=1-3, to=3-3]
	\arrow["{F(f)}"', from=1-6, to=1-8]
	\arrow["{F(g) \circ F(f)}"', from=1-6, to=3-8]
	\arrow["{F(g)}"', from=1-8, to=3-8]
	\arrow["F"', from=4-2, to=4-7]
\end{tikzcd}\]
% \[\begin{tikzcd}
% 	A && {F(A)} \\
% 	\\
% 	B && {F(B)} \\
% 	\\
% 	C && {F(C)} \\
% 	C && {\mathcal{D}}
% 	\arrow["{1_A}"{pos=0.6}, from=1-1, to=1-1, loop, in=150, out=210, distance=5mm]
% 	\arrow["f", from=1-1, to=3-1]
% 	\arrow["{g \circ f}"'{pos=0.6}, curve={height=-24pt}, from=1-1, to=5-1]
% 	\arrow["{1_{F(a)}}", from=1-3, to=1-3, loop, in=150, out=210, distance=5mm]
% 	\arrow["{F(f)}", from=1-3, to=3-3]
% 	\arrow["{F(g) \circ F(f)}", curve={height=-24pt}, from=1-3, to=5-3]
% 	\arrow["g", from=3-1, to=5-1]
% 	\arrow["{F(g)}", from=3-3, to=5-3]
% 	\arrow["F", from=6-1, to=6-3]
% \end{tikzcd}\]

A segunda parte define que existindo uma composição de morfismos em $\mathcal{C}$, também existira uma composição correspondente em $mathcal{D}$.

A terceira parte define que as identidades também são preservadas.

Em algumas partes da matemática porém, os funtores não preservam a ordem dos morfismos. Os funtores que preservam como da regra 1 da parte anterior são chamados de \emph{funtores covariantes}. É interessante também definir os funtores \emph{contravariantes}:

\begin{definition}[Funtor Contravariante, \cite{awodey2010}]
    Um funtor da forma $F : \mathcal{C}^{op} \to \mathcal{D}$ é chamado de \emph{funtor contravarinate} em $\mathcal{C}$. Ou seja, as regras se tornam:
    \begin{enumerate}
        \item Seja $f : A \to B$ em $\mathcal{C}$, então: $F(f : A \to B) : F(B) \to F(A)$
        \item Seja $g \circ f$ em $\mathcal{C}$, então $F(g \circ f) = F(f) \circ F(g)$
        \item $F(1_A) = 1_{F(A)}$
    \end{enumerate}  
\end{definition}

Um exemplo essencial de funtor contravariante são os \emph{pré-feixes}, definidos da seguinte forma:

\begin{definition}[Pré-feixe, \cite{rosiak2022}]
    Um \emph{pré-feixe} (com valor de conjunto) em $\mathcal{C}$, onde $\mathcal{C}$ é uma categoria pequena, é um funtor $\mathcal{C}^{op} \to \textbf{Set}$
\end{definition}


O pré-feixe pode ser visto como uma atribuição de dados locais de acordo com a estrutura de $\mathcal{C}$.

Uma vez definidos funtores, o próximo passo é se perguntar se existe uma categoria que usa funtores como morfismos entre seus objetos, e a resposta é que existe tal categoria, onde objetos são categorias, definida da seguinte forma:

\begin{definition}[\textbf{Cat}, \cite{rosiak2022}]
    A categoria de \emph{categorias pequenas}, denotada por \textbf{Cat}, é a categoria que possui:
    \begin{itemize}
        \item objetos: categorias pequenas
        \item morfismos: funtores entre elas
    \end{itemize}
\end{definition}

Para demonstrar que essa é de fato uma categoria, é necessário definir um morfismo/funtor identidade e a ideia de composição entre morfismos/funtores.

\begin{definition}[\cite{rosiak2022}]
    Dada uma categoria $\mathcal{C}$, o \emph{funtor identidade} é o funtor $id_{\mathcal{C}} : \mathcal{C} \to \mathcal{C}$ que faz o esperado: leva um objeto a ele mesmo e um morfismos a ele mesmo tal que:
    \begin{itemize}
        \item $id_{\mathcal{C}} (c) = c$
        \item $id_{\mathcal{C}} (f) = f$
    \end{itemize}
\end{definition}

Para a composição, sejam $\mathcal{C}$, $\mathcal{D}$ e $\mathcal{E}$ categorias pequenas, e $F : \mathcal{C} \to \mathcal{D}$ e $G : \mathcal{D} \to \mathcal{E}$ dois funtores, a composição de $G$ com $F$, o funtor composição $G \circ F : \mathcal{C} \to \mathcal{E}$, é definido tal que para objetos $c$ em $\mathcal{C}$, $(G \circ F) (c) = G(F(c))$ e para um morfismo $f : c \to c'$ em $\mathcal{C}$ $(G \circ F) (f) = G(F(f))$. Para definir que $G \circ F$ é um funtor é necessário pedir sua \emph{funtorialidade}, ou seja, que ele obedeça às regras impostas na definição de funtores:

(1) Sejam $f, g$ funtores em $\mathcal{C}$ tal que $g \circ f$ também está definido em $\mathcal{C}$, então: 

\begin{equation*}
    \begin{split}
        (G \circ F) (g \circ f) & = G(F(g \circ f))
                                     \\ & = G(F(g) \circ F(f))
                                     \\ & = G(F(g)) \circ G(F(f))
                                     \\ & = (G \circ F) (g) \circ (G \circ F) (f)
    \end{split}
\end{equation*}

onde a primeira e a última linha são a definição da composição de funtores e o méio é a derivação dessa composição.

(2) Seja $c$ qualquer objeto em $\mathcal{C}$, então:

\begin{equation*}
    \begin{split}
        (G \circ F) (1_c) & = G(F(1_C))
                        \\ & = G(1_{F(c)})
                        \\ & = 1_{G(F(c))}
                        \\ & = 1_{(G \circ F)(c)}
    \end{split}
\end{equation*}

A seguir estão alguns exemplos de funtores:

\begin{enumerate}
    \item Ao tratar monoides como categorias por si só, é interessante entender o que seria equivalente a funtores nesse caso. Normalmente, as relações entre monoides são \emph{homomorfismos de monoides}. Sejam dois monoides $(M, e, \cdot)$ e $(N, e', \star)$, um homomorfismo $\phi : M \to N$ é um mapeamento tal que $$\phi(m \cdot m') = \phi(m) \star \phi(m')$$ e $$\phi(e) = e'$$
    É possível ver que $\phi$ possui a estrutura de funtor entre monoides. $\phi$ seria contravariante se $\phi(m \cdot m') = \phi(m') \star \phi(m)$
    \item Existe um funtor $Core : \textbf{Mon} \to \textbf{Grp}$ que pega um monoide e retorna um subconjunto desse monoide que possui elementos inversos, o que faz com que o monoide se torne um grupo, chamado de \emph{cerne} (\emph{Core}) do monoide.
    \item Existe um funtor $U : \textbf{Grp} \to \textbf{Mon}$ chamado de \emph{Forgetful functor} (Funtor esquecido) que pega um grupo e retorna o monoide correspondente, "esquecendo" a estrutura a mais que caracteriza os grupos.
    \item Outro funtor esquecido é $U : \textbf{Cat} \to \textbf{Grph}$ que pega cada categoria e retorna o grafo correspondente a ela. Um Grafo $G = (V, A, s, t)$ é composto de um conjunto de vertices $V$, um conjunto de arestas $A$ que são direcionadas, e um par de funções $s, t : A \to B$ que codifica a direção das arestas ao assinalar a cada aresta $a \in A$ um inicio $s(a) \in V$ e um fim $t(a) \in V$.
\end{enumerate}

A coleção de objetos de uma categoria $\mathcal{C}$ será denotada por $\mathcal{C}_0$ e a coleção de morfismos será denotada por $\mathcal{C}_1$ na próxima definição:

\begin{definition}[\cite{awodey2010}]
    Um funtor $F : \mathcal{C} \to \mathcal{D}$ é dito:
    \begin{itemize}
        \item \emph{injetivo em objetos} se a parte de objetos $F_0 : \mathcal{C}_0 \to \mathcal{D}_0$ é injetiva, é \emph{sobrejetora em objetos} se $F_0$ é sobrejetora
        \item De forma similar, $F$ é \emph{injetiva} (resp. \emph{sobrejetora}) em \emph{setas} se a parte de setas $F_1 : \mathcal{C}_1 \to \mathcal{D}_1$ é injetiva (resp. sobrejetora)
        \item $F$ é dito \emph{fiel} (Faithful) se, para todo $A, B \in \mathcal{C}_0$, o mapa $F_{A, B} : Hom_{\mathcal{C}}(A, B) \to Hom_{\mathcal{D}}(FA, FB)$ definido por $f : F(f)$ é injetivo
        \item Similarmente $F$ é dito \emph{cheio} (full) se $F_{A, B}$ é sempre sobrejetor
    \end{itemize}
\end{definition}

Exemplo: Seja o funtor esquecido $U : \textbf{Grp} \to \textbf{Set}$. $U_0$ é basicamente o mapeamento dos conjuntos que forma o grupo para os próprios conjuntos, logo $U_0$ é injetivo e sobrejetor em objetos. $U_1$ mapeia os homomorfismos de grupo para as funções correspondentes. Mas dois homomorfismos de grupo com o mesmo domínio e codomínio são iguais se são dados pelas mesmas funções nos conjuntos internos. Porém, nem todas as funções em $\textbf{Set}$ são mapeadas por homomorfismos, logo $U_1$ é injetivo em setas mas não sobrejetor em setas. Os motivos para dizer que $U_1$ é injetor em setas podem ser usados para mostrar que $U$ é fiel. 

\subsection{Transformações Naturais}

Uma vez tendo definido morfismos entre categorias, se torna possível pensar morfismos entre esses morfismos. No caso, morfismos entre funtores:

\begin{definition}[Transformação Natural, \cite{rosiak2022}]
    Sejam duas categorias $\mathcal{C}$ e $\mathcal{D}$ e funtores $F, G : \mathcal{C} \to \mathcal{D}$. Uma \emph{Transformação Natural} $\alpha : F \Rightarrow G$ representado em relação a seus dados como: 
    % \[\begin{tikzcd}
    %     C && D
    %     \arrow[""{name=0, anchor=center, inner sep=0}, "F", curve={height=-12pt}, from=1-1, to=1-3]
    %     \arrow[""{name=1, anchor=center, inner sep=0}, "G"', curve={height=12pt}, from=1-1, to=1-3]
    %     \arrow["\alpha", shorten <=3pt, shorten >=3pt, Rightarrow, from=0, to=1]
    % \end{tikzcd}\]
    Consiste no seguinte:
    \begin{itemize}
        \item Para cada objeto $c \in \mathcal{C}$, um morfismo $\alpha_c : F(c) \to G(c)$ em $\mathcal{D}$ chamado de $c$-componente de $\alpha$, a coleção do qual (para todo objeto em $\mathcal{C}$) define os \emph{componentes} da transformação natural
        \item Para cada morfismo $f : c \to c'$ em $\mathcal{C}$ o seguinte quadrado de morfismos, chamado de \emph{quadrado de naturalidade} de $f$, que deve comutar em $\mathcal{D}$:
        \[\begin{tikzcd}
            c && {F(c)} && {G(c)} \\
            \\
            {c'} && {F(c')} && {G(c')}
            \arrow["f"', from=1-1, to=3-1]
            \arrow["{\alpha_{c}}"', from=1-3, to=1-5]
            \arrow["{F(f)}"', from=1-3, to=3-3]
            \arrow["{G(f)}", from=1-5, to=3-5]
            \arrow["{\alpha_{c'}}"', from=3-3, to=3-5]
        \end{tikzcd}\]
    \end{itemize}
    A coleção de transformações naturais entre $F$ e $G$ é por vezes denotada por $Nat(F, G)$
\end{definition}

É dito que morfismos em uma categoria possuem \emph{naturalidade} quando possuem um comportamento parecido com o do \emph{quadrado de naturalidade}, ou seja se $G(f) \circ \alpha_c = \alpha_{c'} \circ F(f)$.

Uma vez que as transformações naturais ajudam a comparar dois funtores entre si, é interessante saber quando os dois funtores são praticamente iguais. Para isso, vamos usar a seguinte definição:

\begin{definition}[Isomorfismo natural, \cite{rosiak2022}]
    Um \emph{isomorfismo natural} é uma transformação natural $\alpha : F \Rightarrow G$ para qual todo componente $\alpha_c : F(c) \to G(c)$ em $\mathcal{D}$ é um isomorfismo (na categoria alvo). Ou seja, cada $\alpha_c$ possui um inverso $\alpha_c^{-1} : G(c) \to F(c)$ onde os inversos formam componentes de uma transformação natural $\alpha^{-1}$ de $G$ para $F$.
\end{definition}

Se $\alpha$ for um isomorfismo, usa-se a notação $\alpha : F \cong G$

Uma vez definida a equivalência entre funtores, é interessante definir equivalência entre categorias:

\begin{definition}[equivalência de categorias, \cite{rosiak2022}]
    Uma \emph{equivalência de categorias} consiste de um par de funtores $F : \mathcal{C} \to \mathcal{D}$ e $G : \mathcal{D} \to \mathcal{C}$ junto com os isomorfismos naturais $\eta : id_{\mathcal{C}} \cong G \circ F$ e $\epsilon : F \circ G \cong id_{\mathcal{D}}$. Outro jeito de dizer isso é que os funtores são inversos entre si "até o isomorfismo natural de funtores". As categorias $\mathcal{C}$ e $\mathcal{D}$ são ditas \emph{equivalentes} se existe uma equivalência de categorias entre elas, isso é denotado por $\mathcal{C} \simeq \mathcal{D}$
\end{definition}

Uma construção interessante é a categoria de setas que possui funtores como objetos e transformações naturais como morfismos, definida como:

\begin{definition}[\cite{rosiak2022}]
    Para qualquer par fixo de categorias $\mathcal{C}$ e $\mathcal{D}$, pode-se formar uma \emph{categoria de funtores} denotada por $\mathcal{D}^{\mathcal{C}}$ (Ou também $Fun(\mathcal{C}, \mathcal{D})$) que possui:
    \begin{itemize}
        \item objetos: todos os funtores de $\mathcal{C}$ para $\mathcal{D}$
        \item morfismos: todas as transformações naturais entre tais funtores
    \end{itemize}
\end{definition}

Para demonstrar o aspecto de morfismo das transformações naturais, é necessário definir a transformação natural identidade, dada simplismente por $id_F : F \Rightarrow F$, e a composição entre transformações naturais, dada pela seguinte definição:

\begin{definition}[\cite{rosiak2022}]
    Sejam $\alpha : F \Rightarrow G$ e $\beta : G \Rightarrow H$ transformações naturais entre os funtores paralelos $F, G, H$ entre $\mathcal{C}$ e $\mathcal{D}$ como no seguinte diagrama:
    % \[\begin{tikzcd}
    %     C &&& D
    %     \arrow[""{name=0, anchor=center, inner sep=0}, "H"', curve={height=30pt}, from=1-1, to=1-4]
    %     \arrow[""{name=1, anchor=center, inner sep=0}, "F", curve={height=-30pt}, from=1-1, to=1-4]
    %     \arrow[""{name=2, anchor=center, inner sep=0}, "G"{description}, from=1-1, to=1-4]
    %     \arrow["\alpha", shorten <=4pt, shorten >=4pt, Rightarrow, from=1, to=2]
    %     \arrow["\beta", shorten <=4pt, shorten >=4pt, Rightarrow, from=2, to=0]
    % \end{tikzcd}\]
    Existe uma transformação natural $\beta \circ \alpha : F \Rightarrow H$, definida em cada componente como: $(\beta \circ \alpha)_c := \beta_c \circ \alpha_c$ dada pela composição de $\beta$ e $\alpha$.
\end{definition}

Esse estilo de composição é denominado de \emph{compisição vertical}

Já a composição horizontal denotada pelo simbolo $\diamond$ dado por $\beta \diamond \alpha : F_2 \circ F_1 \Rightarrow G_2 \circ G_1$, os quais cada componente em $c$ de $\mathcal{C}$ é definido como o composto do seguinte diagrama comutativo:

% \[\begin{tikzcd}
% 	C &&& D && E
% 	\arrow[""{name=0, anchor=center, inner sep=0}, "{G_1}"', curve={height=30pt}, from=1-1, to=1-4]
% 	\arrow[""{name=1, anchor=center, inner sep=0}, "{F_1}", curve={height=-30pt}, from=1-1, to=1-4]
% 	\arrow[""{name=2, anchor=center, inner sep=0}, "{F_2}", curve={height=-30pt}, from=1-4, to=1-6]
% 	\arrow[""{name=3, anchor=center, inner sep=0}, "{G_2}", curve={height=30pt}, from=1-4, to=1-6]
% 	\arrow["\alpha", shorten <=8pt, shorten >=8pt, Rightarrow, from=1, to=0]
% 	\arrow["\beta", shorten <=8pt, shorten >=8pt, Rightarrow, from=2, to=3]
% \end{tikzcd}\]



\end{document}

